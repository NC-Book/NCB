

\exercise{When things go wrong}{2}
We discovered that the iterative procedure for the leading eigenvalue fails in bipartite networks. Let's try it nevertheless.

\subquestion
Consider
\eqn{
{\bf A}=\avecc{0& 1 \\ 1 & 0}
}
and the initial vector
\eqn{
\vec{v}_0 = \avec{1\\2}
}
Use the iteration 
\eqn{
\vec{v}_{i+1} = {\bf A}\vec{v}_i
}
a few times and see what happens.

\solution
Starting from 
\eq{
\vec{v}_0 = \avec{1\\2}
}
we multiply by $\bf A$ to get  
\eq{
\vec{v}_1 = {\bf A} \vec{v}_0 = \avecc{0& 1 \\ 1 & 0}\avec{1\\2} = \avec{2\\1} 
}
and after another multiplication, we are back to 
\eq{
\vec{v}_2 = {\bf A} \vec{v}_1 = \avecc{0& 1 \\ 1 & 0}\avec{2\\1} = \avec{1\\2} = \vec{v}_0 
}
So, this won't get us anywhere. 

\subquestion
Define the shifted matrix ${\bf B}={\bf A}+{\bf I}$ and try again.

\solution
We start again with 
\eq{
\vec{v}_0 = \avec{ 1 \\ 2}  
}
and multiply $\bf B$ to find 
\eq{
\vec{v}_1 = \avec{ 3 \\ 3} \quad \vec{v}_2 = \avec{ 6 \\ 6} 
}
Form now on the entries will just double with every further multiplication. 

\subquestion
Based on the result from (b) state the largest eigenvalue and the eigenvector of $\bf A$. 

\solution
The eigenvector is $\vec{v}=(1,1)^{\rm T}$ or any multiple thereof. We can see from the iterative procedure that the eigenvector gets stretched by a factor two 
in every iteration. So the largest eigenvalue of $\bf B$ is 2, which means that the largest eigenvalue of $\bf A$ is 1.

\subquestion
Bonus: This one is more tricky, but note that in this case the iteration lands us exactly on the eigenvalue in the first step. This actually tells us what the second eigenvalue of $\bf A$ is. Explain!

\solution
Our initial vector was not the leading eigenvector, so it contained some component in the direction of the second eigenvector. But whatever, this component was it vanished in the first multiplication, which means it was multiplied by zero. So zero must be the second eigenvalue of $\bf B$, which means that the second eigenvalue of $\bf A$ is -1. 

