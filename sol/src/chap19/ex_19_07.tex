\exercise{Traffic: Betweenness centrality}{}
Consider the traffic a final time, and this time rank the nodes by betweenness centrality. (If there is multiple alternative shortest paths between nodes assign importance proportionally, e.g.~half a point if there are two paths.)  Is the result reasonable?

\solution
Thankfully, we can reuse the table from (c). We just need to count occurrences of intermediate nodes. The challenge is to not let the bookkeeping get out of hand. For this purpose I made another table

\begin{center}
\begin{tabular}{c c c c}
Node & On Paths (no alternative) & On Path (one alternative) & Centrality \\\hline
A & B-X,C-X,D-X,E-X,F-X,G-X,H-X & - & 7 \\
B & A-D,A-G,C-D,C-G,D-X,G-X & A-F,A-H,F-X,H-X & 8 \\
C & A-E,B-E,E-X & A-F,A-H,F-X,H-X & 5 \\
D & B-H & A-H,F-X,G-H,H-X & 3 \\
E & C-F,C-H & A-F,A-H,F-X,H-X & 4 \\
F & E-G & G-H & 1.5 \\
G & B-F & A-F,D-F & 2   \\
H & D-E & D-F & 1.5 \\
X & - & - & 0 \\  
\end{tabular}
\end{center}
Again we end up with B as the most important node and X as the least important which makes sense. A surprise is maybe the relatively high value for A. This highlights A as a bottleneck. Generally, betweenness centrality is used when we are interested in finding such bottlenecks. 