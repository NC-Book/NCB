
\exercise{Reactivity}{4}
Reactivity was introduced by Neubert and Caswell in 1997 to measure the initial response of a dynamical system to perturbations. Specifically, reactivity is defined as the initial amplification of a perturbation by the dynamics. In this exercise we will re-derive reactivity, interpret it as a bespoke centrality for a given question and also discuss its role in motifs. 

\subquestion
We are interested in the relative rate of growth $a$ of the size of small perturbation $\vec{\delta}$ to a steady state, that is 
\eqn{
a = \frac{1}{|\vec{\delta}|} \frac{{\rm d}}{{\rm d} t} |\vec{\delta}|
}
Recall that $|\vec{\delta}|=\sqrt{\vec{\delta}^{\rm T}\vec{\delta}}$, and $\dot{\vec{\delta}}={\bf J}\vec{\delta}$. Then show 
\eqn{
  a = \frac{\vec{\delta}^{\rm T} {\bf H} \vec{\delta}}{\vec{\delta}^{\rm T}\vec{\delta}},
}
where ${\bf H} = ({\bf J}+{\bf J}^{\rm T})/2$ is the symmetric part of the Jacobian. 

\solution
We will need the transposed version of our lineraized dynamics
\eq{
   \dot{\vec{\delta}}^{\rm T} = {\vec{\delta}}^{\rm T} {\bf J}^{\rm T},
}
then we compute
\eqa{
\frac{1}{|\vec{\delta}|} \frac{{\rm d}}{{\rm d} t} |\vec{\delta}| &=& \frac{1}{|\vec{\delta}|} \frac{{\rm d}}{{\rm d} t} \sqrt{\vec{\delta}^{\rm T}\vec{\delta}} \\
&=& \frac{1}{|\vec{\delta}|} \frac{1}{2\sqrt{\vec{\delta}^{\rm T}\vec{\delta}}} \frac{\rm d}{{\rm d} t} \vec{\delta}^{\rm T}\vec{\delta} \\
&=& \frac{1}{2|\vec{\delta}|^2} \left( \dot{\vec{\delta}}^{\rm T}\vec{\delta} + \vec{\delta}^{\rm T}\dot{\vec{\delta}} \right) \\
&=& \frac{1}{2|\vec{\delta}|^2} \left( \vec{\delta}^{\rm T} {\bf J}^{\rm T} \vec{\delta} + \vec{\delta}^{\rm T}{\bf J}\vec{\delta} \right) \\
&=& \frac{\vec{\delta}^{\rm T} {\bf H} \vec{\delta}}{\vec{\delta}^{\rm T}\vec{\delta}},
}
which is the desired result. This type of function is also called a Rayleigh quotient.

\subquestion 
Consider the predator prey system 
\eqan{
\dot{X} &=& X - XY/(3+X) \\
\dot{Y} &=& 2XY/(3+X) - Y.
}
Compute $\bf J$ and $\bf H$ in the nontrivial steady state.

\solution
The second equation gives us the steady state condition
\eq{
0 = \frac{XY}{3+X} - Y \\
}
dividing by $Y$ yields 
\eqa{
0 &=& 2X/(3+X) -1 \\
3+X &=& 2X \\
3 &=& X.
}
So now that we know that $X=3$, we can compute $Y$ from the first row:
\eqa{
 0 &=& X(4-X) - 2XY/(3+X) \\
 0 &=& 3 - Y\\
 Y &=& 3.
}
So the non-trivial steady state is at $(X^*,Y*)=(3,3).$ Now we compute 
\eqa{
J_{11} &=& \left.\frac{\partial}{\partial X} \dot{X} \right|_* = \left. 4-2X - \frac{2Y}{3+X} + \frac{2XY}{(3+X)^2}\right|_* =  4-6 -1+ 18/36 = -2.5  \\
J_{12} &=& \left.\frac{\partial}{\partial Y} \dot{X} \right|_* = \left. - \frac{X}{3+X} \right|_* = -0.5 \\
J_{21} &=& \left.\frac{\partial}{\partial X} \dot{Y} \right|_* = \left.  \frac{2Y}{3+X} - \frac{2XY}{(3+X)^2}  \right|_*  = 1 - 18/36 = 0.5 \\
J_{22} &=& \left.\frac{\partial}{\partial Y} \dot{Y} \right|_* = \left. \frac{2X}{3+X} - 1 \right|_* = 6/6-1 = 0  
}
Hence the Jacobian is 
\eq{
{\bf J} = \avecc{-2.5 & -0.5\\ 0.5 & 0}  
}
and we can compute 
\eq{
{\bf H} = \frac{\avecc{-2.5 & -0.5\\ 0.5 & 0} + \avecc{-2.5 & 0.5\\ -0.5 & 0}}{2} = \avecc{-2.5 & 0 \\ 0 & 0}
}

\subquestion 
We define the reactivity of the system as the largest amplification $a$ that can be observed in response to any small perturbation. Because $\bf H$ is a symmetric (hermitian) matrix, it's eigenvectors are orthogonal. Hence they can be normalized such that $\vec{v_n}^{T}\vec{v_m} = \delta_{nm}$, where $\delta$ is the Kronecker delta. Use this to show that the maximal value of $a$ is observed when the perturbation is in the direction of the eigenvector of $\vec{v}$ with the largest eigenvalue.  

\solution 
we start with 
\eq{
  a = \frac{\vec{\delta}^{\rm T} {\bf H} \vec{\delta}}{\vec{\delta}^{\rm T}\vec{\delta}}
}
and then decompose $\vec{\delta}$ into eigenvectors such that 
\eq{
\vec{\delta} = \sum c_n \vec{v_n}, 
}
where the $C_n$ are coefficients, chosen such that the equation is satisfied for the $\vec{\delta}$ under consideration. Likewise, 
\eq{
\vec{\delta}^{\rm T} = \sum c_n \vec{v_n}^{\rm T}, 
}
Substituting the expansion we find 
\eq{
  a = \frac{\left(\sum c_n \vec{v_n}^{\rm T}\right)  {\bf H} \left(\sum c_n \vec{v_n}\right)}{\left(\sum c_n \vec{v_n}^{\rm T}\right)\left(\sum c_n \vec{v_n}\right)}. 
}
Pulling the $\bf H$ into the sum on it's right and using the eigenvector equation simplifies this to 
\eq{
  a = \frac{\left(\sum c_n \vec{v_n}^{\rm T}\right) \left(\sum \lambda_n c_n \vec{v_n}\right)}{\left(\sum c_n \vec{v_n}^{\rm T}\right)\left(\sum c_n \vec{v_n}\right)}. 
}
Now let's consider the denominator of this fraction. To avoid confusion, we now call index in the second sum $m$. Then the denominator reads
\eq{
\left(\sum_n c_n \vec{v_n}^{\rm T}\right)\left(\sum_m c_m \vec{v_m}\right) = \sum_{n,m} c_n c_m  \vec{v_n}^{\rm T} \vec{v_m} = \sum_{n,m} c_n c_m \delta_{nm} 
}
Clearly the only non-zero terms in the sum will now be the ones where $n=m$ so we can use our vanishing-act summation rule to simplify the denominator to 
\eq{
\sum_{n,m} c_n c_m \delta_{nm}  = \sum (c_n)^2
}
Doing the same for the numerator yields 
\eq{
\left(\sum c_n \vec{v_n}^{\rm T}\right) \left(\sum \lambda_n c_n \vec{v_n}\right) = \sum (c_n)^2 \lambda_n. 
}
Substituting back we arrive at 
\eq{
a = \frac{\sum (c_n)^2 \lambda_n }{\sum (c_n)^2}
}
This equation has the form of a weighted average over the $\lambda_n$, we can write it as 
\eq{
a = \sum w_n \lambda_n 
}
where the weights are 
\eq{
w_n=\frac{{c_n}^2}{\sum_m {c_m}^2}.
}
As is all proper weighted averages the weights are positive and sum to 1.  To maximize the a weighted average, the entire weight must be placed on the highest term that we average over. Since we average over the $\lambda_n$ in this case. The highest term is the highest eigenvalue, let's call it $\lambda_1$. Hence we set the corresponding $c_1=1$ and $c_n=0$ for all $n\neq 1$. Now the amplification simplifies to  
\eq{
a = \lambda_1. 
}
We have shown that the reactivity of a system is identical to the largest eigenvalue of the symmetric (hermitian) part of it's Jacobian matrix. 

\subquestion
A system is said to be reactive if it has positive reactivity, $a>0$. Is our predator-prey system from above reactive. Furthermore, would a system with the following Jacobian be reactive:
\eqn{
{\bf J} = \avecc{-8 & -8 \\ 2 & 0}.
}

\solution
For the predator-prey example we already know the matrix $\bf H$. We now need to solve the eigenvector equation 
\eq{
\avecc{-2.5 & 0 \\ 0 & 0} \vec{v_n} = \lambda_n \vec{v_n}. 
}
Because the matrix is diagonal we can directly read off the eigenvalues,
\eq{
\lambda_1 = 0, \qqq \lambda_2 = -2.5 .
}
So the largest eigenvalue is 0, if we interpret the statement from the question strictly that makes the system non-reactive. Not even the worst possible perturbations grows initially exponentially. As the zero is just on the edge of what we consider reactive, we could also decide that we want to be more careful and check the next order of the Taylor expansion around the steady state for potential non-exponential growth of perturbations. 

For the second example we compute the symmetric part of the Jacobian
\eq{
{\bf H} = \avecc{-8 & 3 \\ 3 & 0}
}
and solve the characteristic polynomial
\eqa{
\lambda^2+8\lambda -9 &=& 0 \\
(\lambda+4)^2 -9 - 16 &=& 0 \\
(\lambda+4)^2 &=& 25 \\
\lambda+4 &=& \pm \sqrt{25} \\
\lambda &=& -4 \pm 5. \\
}
Hence the eigenvalues are $\lambda_1=1$ and $\lambda_2=-9$. Since $\lambda_1>0$, the system is reactive. The worst possible small perturbation will initially grow like $\exp{\lambda_1 t}=\exp{t}$. 

\subquestion
The reactivity of a system is at least as large as the reactivity found in any motif within the system. Explain why. 

\solution
Suppose we have a large system, with a correspondingly large Jacobian matrix. To compute reactivity we need the symmetric part of the Jacobian $\bf H$. To compute the reactivity we within a motif we then consider the sub-matrix (the minor) $\bf S$ corresponding to set of nodes within the motif. Because $\bf H$ is symmetric Cauchy's interlacing theorem implies that the largest eigenvalue of $\bf H$ is greater or equal to the largest eigenvalues of $\bf S$. Hence the system's reactivity must be greater or equal to the reactivity computed within $\bf S$. 

This is a very useful insight because it allows to obtain bounds for the reactivity of systems observed in nature, without monitoring all of their variables.  


