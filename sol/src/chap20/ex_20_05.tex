
\exercise{Square proof}{5}
If $\lambda_1,\ldots,\lambda_N$ are the eigenvalues of an adjacency matrix of a network containing $N$ nodes and $K$ links then 
\eq{
\frac{1}{2}\sum_n {\lambda_n}^2 = K.
}

\subquestion
Check that the statement above is correct for some networks of your choice. 

\solution
Let me first try this for a network containing a single node and no links. The adjacency matrix is 
\eq{
{\bf A} = (0)
}
Clearly the only eigenvalue is 0 and the statement is trivially true. I can now see that the same would happen without links. So lets try a linked pair next. The adjacency matrix is 
\eq{
{\bf A}=\avecc{ 0  & 1 \\ 1 & 0 } 
}
which has the eigenvalues $\lambda_1=1$ and $\lambda_2=-1$ and hence 
\eq{
\frac{1^2+-(-1)^2}{2} = 1 
}
which is the number of links in the network. We can also try a three-node chain. The eigenvalues in this case are $\sqrt{2}$, $-\sqrt{2}$, and $0$, and hence
\eq{
\frac{0+2+2}{2} = 2,
}
so this seems to work!

\subquestion 
Prove that the statement holds for any simple graph. 

\solution
This is very easy once we have the right idea (and hard otherwise). We know form the lecture that the degrees of the nodes appear on the diagonal of ${\bf A}^2$. The trace is the sum of the diagonal terms and hence 
\eq{
{\rm Tr}( {\bf A}^2) =2K
}
We also know that the trace of a matrix is the sum of its eigenvalues. So let's see if we can express the eigenvalues of ${\bf A}^2$ as a function of the eigenvalues of $\lambda_n$ of $\bf A$. If $\lambda_n$ is an eigenvalue of $\bf A$ there must be a vector $\vec{n_n}$ such that 
\eq{
{\bf A}\vec{v_n} = \lambda_n \vec{V_n}
}
We might have the suspicion that this vector is also an eigenvector of ${\bf A}^2$. Let's try 
\eq{
{\bf A^2}\vec{v_n}={\bf AA}\vec{v_n} = {\bf A}\lambda_n \vec{v_n}={\lambda_n}^2\vec{v_n}
}
Since we can do this for every eigenvector the eigenvalues of $\kappa_n$ of ${\bf A}^2$ are
\eq{
\kappa_n = {\lambda_n}^2
}
Thus we can write
\eq{
2K = {\rm Tr}({\bf A}^2) = \sum_n \kappa_n = \sum_n \lambda_n
}
