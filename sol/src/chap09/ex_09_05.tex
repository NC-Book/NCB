
\exercise{A quick test}{2}
Consider a network where every node has degree 1. In this network it is not very hard to guess what the distribution of component sizes looks like. 

\subquestion Construct the generating function $G$ of the degree distribution and use it to compute the generating function $Q$ of the excess degree distribution. 

\solution
The generating function is simply 
\eq{
G=x
}
We compute the excess degree generating function as
\eq{
Q=\frac{G'}{G'(1)}=1
}

\subquestion Use the equation from the lecture, $Y=xQ(Y)$, to determine the generating function $Y$ that generates the number of nodes in a branch. 

\solution
We substituting our $Q=1$ into $Y=x(Q(Y))$ gives us 
\eq{
Y=x
}
This is the generating function's way of telling us that every branch will contain exactly one node. 

\subquestion Use $C=xG(Y)$ to find the function $C$ that generates the component size distribution and explain the the result.

\solution
We compute 
\eq{
C=xG(Y)=x^2
}
This is the generating function of a random process that returns the result 2 with 100\% probability. This is the expected result: A network where every node has degree 1 must exist entirely of pairs of two nodes. SO each component has size 2. 