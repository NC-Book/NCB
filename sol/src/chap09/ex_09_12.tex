\exercise{Attacks on networks}{4}
In the next chapter we study attacks on networks. In this exercise we can explore some relevant calculations already. Let us consider a network where every node has initially degree 4. Then an attack removes half the links at random.  

\subquestion Construct the degree distribution after the attack, and use it to compute the mean excess degree after the attack without using generating functions 

\solution
Without generating functions this is a bit tedious. We have to work out the probabilities for all of the different scenarios: The probability that a node survives with all links intact is $1/16$, the probability that a node loses exactly one link is $4/16$ (see Chap.~7), and so on. This leads to the following degree distribution
\eq{
p_k = \frac{1}{16} (\delta_{k,0}+4\delta_{k,1}+6\delta_{k,2}+4\delta_{k,3}+\delta_{k,4})   
}
When we remove half the links from the network we expect the degree to halve as well (remember $z=2K/N$? If we reduce $K$, $z$ shrinks proportionally) 
To verify we can compute

\eqa{
z&=&\sum k p_k \\
&=& \sum \frac{1}{16} (k\delta_{k,0}+4k\delta_{k,1}+6k\delta_{k,2}+4k\delta_{k,3}+k\delta_{k,4}) \\
&=& \frac{1}{16} (4+6\cdot 2+4\cdot 3+4) = \frac{32}{16}=2.  
}
We can now compute the excess degree distribution in the usual way 
\eqa{
q_k&=&\frac{(k+1)p_{k+1}}{z} \\
   &=& \frac{1}{2}\frac{1}{16} (4\delta_{k+1,1}+12\delta_{k+1,2}+12\delta_{k+1,3}+4\delta_{k+1,4}) \\
   &=& \frac{1}{8} (\delta_{k,0}+3\delta_{k,1}+3\delta_{k,2}+1\delta_{k,3}) \\
}
Finally, we compute the mean excess degree,
\eqa{
q&=&\sum k q_k \\
 &=&\frac{1}{8} \sum  (k\delta_{k,0}+3k\delta_{k,1}+3k\delta_{k,2}+1k\delta_{k,3}) \\
 &=&\frac{1}{8} (3+6+3) = \frac{12}{8}= 1.5
}
This wasn't horrible, but it certainly wasn't an elegant calculation either. 

\subquestion Now we do the same calculation with generating functions: Write the generating function $G$ of the degree distribution before the attack. Also write the generating function $A$ for a coin flip that gives us a result of 0 or 1 with equal probability. Use $G$ and $A$ to write the generating function $G_{\rm a}$ for the degree distribution after the attack. Finally, use this to construct $Q_{\rm a}$ the generating function for the excess degree distribution after the attack and use it to compute the mean excess degree after the attack. (the calculation will be short)

\solution
This is sooooo much nicer. The generating function before the attack is 
\eq{
G=x^4.
}
The generating function for a coin flip that results in 0 or 1 is
\eq{
A=\frac{1}{2}+\frac{x}{2}.
}
To find the generating function after the attack we can think along the following lines. If we pick a node at random then the number of links that the node had in the original network is described by $G$. However, since some of the links may have been destroyed in the attack we need to flip a coin for each link to see if it still there and then sum over the results of the coin flips. This scenario is covered by our dice-of-dice rule so the result is 
\eq{
G_{\rm a}=G(A)=\left(\frac{1}{2}+\frac{x}{2}\right)^4 
}
We can now compute the excess degree distribution after the attack using the convenient formula from this chapter
\eq{
Q_{\rm a}=\frac{G'_{\rm a}}{G'_{\rm a}(1)} 
}
We compute 
\eq{
G'_{\rm a}=  4\left(\frac{1}{2}+\frac{x}{2}\right)^3 \frac{1}{2} = 2 \left(\frac{1}{2}+\frac{x}{2}\right)^3
}
which reveals the mean degree after the attack $G'_{\rm a}(1)=2$ and furthermore 
\eq{
Q_{\rm a}= \left(\frac{1}{2}+\frac{x}{2}\right)^3.
}
We can now compute the mean excess degree after the attack
\eq{
q=Q_{\rm a}'(1) = 3\left(\frac{1}{2}+\frac{1}{2}\right)^2 \frac{1}{2} = 1.5
}
This calculation seems almost magical: At first it is hard to understand why we do these specific steps, but once we have gained this understanding we can do great things with little effort. 

\solutionend
