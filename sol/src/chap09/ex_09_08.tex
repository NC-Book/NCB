\exercise{Series expansions}{3}
\label{exGeneratingExSeries}
Find the series expansions of the following functions
\begin{center}
a) $\exp{x}$\qqq  b) $1/(1+x)$\qqq c) $1/(1-x)$ 
\end{center}

\solution
To find the series expansion we need to find all the Taylor coefficients. There is an infinite number of them, but because maths is so nicely (and massively) parallel we can find all of them in one go. 

Let's start with part (a). As ingredients for the solution we need the derivatives of the function 
\eq{
f(x)=\exp{x}
}
Fortunately,
\eq{
\partial_x \exp{x} = \exp{x}.
}
No matter how many times we differentiate $\exp{x}$, the result is always $\exp{x}$ itself. 

The second step is to use our formula for the Taylor coefficients,
\eq{
c_k = \frac{1}{k!}f^{(k)}(0).
}
We can now substitute the known derivatives
\eq{
f^{(k)}(x)=\exp{x}
}
to find
\eq{
c_k=\frac{1}{k!} \exp{0} = \frac{1}{k!}.
}
Hence the series expansion is 
\eq{
\exp{x} = \sum c_k x^k = \sum \frac{x^k}{k!},
}
of course, by now, this expansion should be an old friend of yours, but good to know that we can just derive again should we ever forget it. 

For part (b) the derivatives are a little bit more complicated. We compute the first few to get an idea. The easiest way to do this is to write the function as 
\eq{
f(x)=\frac{1}{1+x} = (1+x)^{-1}.
}
Now we compute 
\eqa{
f(x)&=&(1+x)^{-1}\\
f^{(1)}(x)&=& -(1+x)^{-2} \\ 
f^{(2)}(x)&=& +2 (1+x)^{-3}\\ 
f^{(3)}(x)&=& -6 (1+x)^{-4} \\ 
}
So what's happening here? Every time we differentiate the exponent on $(1+x)$ decreases by 1. Additionally on the $k$th differentiation the whole function is multiplied by $-k$. Multiplying up these factors should give us a factorial except that the sign also changes in every step. Actually, we can think of this changing sign as a factor $(-1)$ being multiplied in every differentiation. Putting all of these things together reveals a pattern,
\eq{
f^{(k)}(x)= (-1)^k k! (1+x)^{-(k+1)}
}

For the purpose of this exercise spotting the pattern in this way is all we need. If we were doing proper maths, we would also want to prove that this is indeed correct. So just in case you want to do this, you can do a proof by induction. (If not just skip ahead to the next paragraph.) We start by proving that our identified pattern is correct for $n=0$, which is straight forward
\eq{
f^{0}(x)=(-1)^0 0! (1+x)^{-(0+1)} = (1+x)^{-1}
}
which we know to be true. Now we prove that if our pattern holds for some $k$ then it must also hold for $k+1$. We do this by assuming(!) that 
\eq{
f^{(k)}(x)= (-1)^k k! (1+x)^{-(k+1)}
}
is true and then compute $f^{(k+1)}$ by differentiating, which leads to 
\eqa{
f^(k+1)&=&\partial_x (-1)^k k! (1+x)^{-(k+1)} \\
  &=& (-1)^k k! \partial_x (1+x)^{-(k+1)} \\ 
  &=& (-1)^k k! (-(k+1)) (1+x)^{-(k+2)} \\ 
  &=& (-1)^k k! (-1)(k+1) (1+x)^{-(k+2)} \\ 
  &=& (-1)^(k+1) (k+1)! (1+x)^{-(k+1+1)}  
}
We can now see that the result for $k+1$ conform to the same pattern that we have identified. to make this clearer we can introduce $\hat{k}=k+1$ and write the result as 
\eq{
f^{(\hat{h})} = (-1)^{\hat{k}} \hat{k}! (1+x)^{-(\hat{k}+1)}.
}
So, what have we actually shown? We know that our pattern is correct for $k=0$. We also proved that if it is correct for some $k$ it is also correct for $k+1$, so the fact that it's correct for $k=0$ implies that it must also be correct for $k=1$, and that in turn implies that it must be correct for $k=2$. and so on. The whole chain unravels to infinity. We have proved that the identified pattern is correct. 

If you skipped the previous paragraph, welcome back. If not I hope you enjoyed this little excursion. In any case we now know the pattern that the derivatives of $f$ follow and we can plug them into our coefficient formula,
\eq{
c_k = \frac{1}{k!}f^{(k)}(0) = \frac{1}{k!} (-1)^k k! (1+0)^{-(k+1)} = (-1)^k 
}
It's nice when things simplify like that, isn't it. With these coefficients the series expansion is just 
\eq{
f(x)=\frac{1}{1+x}=\sum c_k x^k = \sum (-1)^k x^k = 1-x+x^2-x^3+\ldots 
}
as we claimed in the chapter text.

Finally part (c) is very similar to part (b), we write 
\eq{
f(x)=\frac{1}{1-x} = (1-x)^{-1}.
}
Now we compute 
\eqa{
f(x)&=&(1+x)^{-1}\\
f^{(1)}(x)&=& (1-x)^{-2} \\ 
f^{(2)}(x)&=& 2 (1+x)^{-3}\\ 
f^{(3)}(x)&=& 6 (1-x)^{-4} 
}
In this case the pattern is simpler than in part (b) 
\eq{
f^{(k)}(x)= k! (1+x)^{-(k+1)}
}
We won't redo the introduction proof for this case (it works exactly as in b). Once we have convinced ourselves that the identified pattern does indeed hold, we compute the Taylor coefficients 
\eq{
c_k = \frac{1}{k!}f^{(k)}(0) = 1, 
}
so in this case the series is simply
\eq{
f(x)=\frac{1}{1+x}=\sum c_k x^k = \sum x^k = 1+x+x^2+x^3+\ldots 
}