
\exercise{Solving series}{3}
Use generating functions to find the solutions of $\sum c_k$, where
$c_0=1$ and 
\eqn{
\mbox{a) } c_{k+1}=\frac{c_k}{2} \qqq \mbox{b) } c_{k+1}=-c_k/3+1  
}
Hint: You might want to use the result of Ex.~9.6

\solution 
For part (a) we define the generating function
\eq{
G=\sum c_k x^k
}
We can write the iteration rule in the form 
\eq{
c_k=2c_{k+1}. 
}
Substitution into $G$ yields 
\eq{
G= \sum 2c_{k+1} x^k. 
}
We shift the index of the sum, which yields
\eq{
G = -\frac{2c_0}{x} + \sum 2 c_k x^{k-1}.
}
To make the right-hand side more similar to the definition of $G$ we write it as  
\eq{
G=\frac{2}{x}\left(-c_0 + \sum c_k x^k \right).
}
We can now use the definition of $G$ and $c_0=1$ to find the self-consistency condition
\eq{
G=\frac{2}{x}(G-1),
}
which we can solve for
\eq{
G=\frac{2}{2-x}.
}
We can now evaluate
\eq{
\sum c_n = G(1)=\frac{2}{2-1} = 2,
}
which is the desired result. 

For part (b) we start again by defining 
\eq{
G=\sum c_k x^k 
}
and solving the iteration rule for $c_k$,
\eq{
c_k=3(1-c_{k+1}).
}
Substitution into $G$ yields
\eq{
G=\sum 3(1-c_{k+1}) x^k.
}
It is useful to split this sum up and pull the factor of 3 out. 
\eq{
G= 3 \left( \sum x^k \right) - 3 \left( \sum c_{k+1} x^k \right)
}
We know from Ex.~9.8 that the term in the first bracket is the series expansion of $1/(1-x)$. To deal with the second term we start again with the index shift
\eq{
G= \frac{3}{1-x} - 3 \left( \sum c_{k} x^{k-1} \right) + \frac{3}{x}c_0. 
}
After pulling $1/x$ from the sum we have
\eq{
G= \frac{3}{x} \left (\frac{x}{1-x}+c_0 -  \sum c_{k} x^{k} \right) .
}
We can now use $c_0=1$ and identify $G$, which yields
\eq{
G= \frac{3}{x} \left (\frac{x}{1-x}+1- G \right).
}
The 1 actually combines beautifully with the fraction that precedes it, so that 
\eq{
G= \frac{3}{x}\left(\frac{1}{1-x}- G \right).
}
Isolating $G$, we can write
\eq{
(x-3)G=\frac{3}{1-x} 
}
and hence the result 
\eq{
G=\frac{3}{(x-3)(1-x)}.
}
So, $G(1)=\infty$, this is a divergent series. 
\solutionend
