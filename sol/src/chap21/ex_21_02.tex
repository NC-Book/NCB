\exercise{Spectral methods in the wild}{3}
A supermarket chain approaches you. Their goal is to identify different groups in their customer base. They have a large dataset that containing 20 million customers, for each customer the dataset describes which products they have bought in the past. Describe in bullet points how you would approach this question. 

\solution

\begin{itemize}
\item We can represent the customers previous purchases as a table. This table could state the rate at which the customer buys an item or just whether the customer has ever bought the item. For example it could be something line this
\begin{center}
\begin{tabular}{l  c c c c}
 & \sw{Product 1} & \sw{Product 2} &  \sw{Product 3} & $\cdots$ \\\hline
 Customer 1 & 1 & 0 & 0 & \ldots \\
 Customer 2 & 1 & 1 & 0 & \ldots \\
 Customer 3 & 0 & 0 & 1 & \ldots \\
 Customer 4 & 1 & 1 & 0 & \ldots \\
 $\vdots$
\end{tabular}
\end{center}

\item We standardize the products such that every column in the table has mean zero and variance one
\item Now we regard the rows as vectors and define a distance $d_{ij}$ between customers as the euclidian distance between their respective vectors
\item  We define a similarity $s_{ij}=1/d_{ij}$.
\item We threshold the similarities. For example a similarity $s_{ij}$ is retained if it is among the top-10 similarity scores for either customer $i$ or $j$. All other  $s_{ij}$ are set to zero. 
\item  We now regard the $s_{ij}$ as elements of a weighted adjacency $\bf s$ and construct the corresponding Laplacian ${\bf L}=-{\bf s}+{\bf D}$, where ${\bf D}=\delta_{ij} \sum_k s_{ik}$. 
\item We compute the second smallest eigenvalue of $\bf L$ and the corresponding eigenvector $\vec{v}$.
\item The customers who correspond to a positive entry in the eigenvector are identified as a cluster, the customers who correspond to a negative entry in the eigenvector form another cluster. 
\item Additional eigenvectors can be used to split the clusters further. 
\end{itemize}
