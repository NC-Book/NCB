\exercise{Hypergraph components}{4}
Hypergraphs are networks, where links connect more than two nodes. 
(Imagine them as multi-way connectors). Compute the giant component size in a hypergraph where half of the nodes have degree 1, half the nodes have degree 3 and each link connects exactly 3 nodes.

\solution
We follow the same approach as in the chapter, however for every link the node now has two chances to connect to the giant component.
So we write
\eqa{
s=1-\sum p_k v^{2k}
v=\sum q_k v^{2k}
}
We also know the degree distribution 
\eq{
p_k = \frac{1}{2}\delta_{k,1}+\frac{1}{2}\delta_{k,3}
}
and we can therefore compute the mean degree $z=2$ and the excess degree distribution
\eq{
q_k=\frac{(k+1)p_{k+1}}{z} = \frac{1}{4} \delta_{k,0} + \frac{3}{4}\delta_{k,2}  
}
Substituting into the self-consistency condition for $v$ yields
\eq{
v=\frac{1}{4}+\frac{3}{4}v^4 
}
This suggests that $v\approx 1/4$, indeed iteration reveals 
$v\approx 0.2530765865$. This is close enough that I am happy to continue with $v=1/4$. We now substitute into the equation for the giant component size
\eqa{
s&=&1-\frac{1}{2}v^2 - \frac{1}{2}v^6 \\
 &=&1-\frac{1}{2}\frac{1}{4^2}-\frac{1}{2}\frac{1}{4^6}  \\
 &=&1-2^{-5}-2^{-13}=0.9686
}
Using the numerical value from the iteration instead yields $0.9678$. 

Predictably the three-way connectors are pretty good and holding this network together, leading to a giant component that spans 96\%
of the network. 