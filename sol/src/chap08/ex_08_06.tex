\exercise{Sexual contact network}{3}
Extensive surveys found that typical people have had 5 sexual partners. 

\subquestion Does the human sexual contact network have a giant component? And, if yes what proportion of people would you expect to be in the giant component, assuming that the network is an ER random graph? 

\solution
With mean degree $z=5$, we expect a giant component. (Also sexually transmitted diseases would not survive if there wasn't one.). 

To estimate the size of the giant component we start at an initial guess of $s_0=0.5$ and iterate 
\eq{
s_{i+1}= 1-\exp{-5s_i}
}
The iteration yields 
\begin{center}
0.9179150014,
0.9898428258,
0.9929110222,
0.9930189442,
0.9930227102,
0.9930228416,
0.9930228462,
0.9930228463,
0.9930228463
\end{center}
So even at $z=5$ more than 99\% of people end up in the giant component. 

\subquestion 
The real sexual contact network is more heterogeneous. How would your estimate of the giant component size change if we assumed the degree distribution $p_k=0.8 \delta_{k,1} + 0.2 \delta_{k,21}$?
(Bonus: What is your expectation of the number of contacts of your partner)

\solution
In this case the first step is to compute the probability that a random link does not lead to a node in the giant component, $v$. From the chapter we know 
\eq{
\label{eqExV}
v=\sum q_k v^k.
}
For this we need the the excess degree distribution $q_k$. 
Using results from Chap.~6 we compute
\eqa{
q_k &=& \frac{(k+1)p_{k+1}}{z} \\
    &=& \frac{1}{5} \left( (k+1) 0.8 \delta_{k+1,1} + (k+1) 0.2 \delta_{k+1,21} \right) \\
    &=& \frac{1}{5} \left( 0.8 \delta_{k+1,1} + 21\cdot 0.2 \delta_{k+1,21} \right) \\
    &=& \frac{1}{5} \left( 0.8 \delta_{k,0} + 21\cdot 0.2 \delta_{k,20} \right) \\
    &=& \frac{4}{25} \delta_{k,0} + \frac{21}{25} \delta_{k,20} 
}
Substituting into the self-consistency condition yields
\eqa{
  v&=&\sum  \left( \frac{4}{25} \delta_{k,0} + \frac{21}{25} \delta_{k,20} \right) v^k \\
   &=& \frac{4}{25} + \frac{21}{25} v^{20}
}
We can solve the resulting equation 
\eq{
   v = \frac{4}{25} + \frac{21}{25} v^{20}
}
by iteration. However, in this case there is also another way. Since it is clear that $v<1$ (it's a probability), $v^{20}$ will probably be a very small number, so let's assume it's 0. (Try the iteration if you don't trust me.) Hence we find the result 
\eq{
v=\frac{4}{25}
}
We can now compute the giant component size as 
\eqa{
s&=&1-\sum p_k v^k \\
 &=&1 - \sum \left(0.8 \delta_{k,1} + 0.2 \delta_{k,21} \right)v^k \\
 &=&1 - \sum \left(0.8 v^k \delta_{k,1} + 0.2 v^k \delta_{k,21} \right) \\
 &=&1 - \sum \left(0.8 v^1 \delta_{k,1} + 0.2 v^{21} \delta_{k,21} \right) \\
 &=&1 - \left(0.8 v + 0.2 v^{21}\right) \\ 
 &\approx& 1 - 0.8 v \\
 &\approx& 1 - \frac{4}{5} \frac{4}{25} = \frac{109}{125}.\\
}
We can also write this as 87.2\%. This is still large but we can see that the heterogeneous degree distribution leads to a somewhat smaller giant component. We will come back to this insight in Chap.~10.

The bonus question is actually asking for the mean excess degree. Since we already know the excess degree distribution we compute
\eqa{
q&=&\sum k q_k \\ 
&=&\sum k \left( \frac{4}{25} \delta_{k,0} + \frac{21}{25} \delta_{k,20} \right) \\
&=&\sum  20  \frac{21}{25} \delta_{k,20} \\
&=&   \frac{420}{25} = 16.8,
}
which is a different way of saying that condoms are a great invention ;o) 