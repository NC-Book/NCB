\exercise{Abstract giant component\label{exCowsabstractGC}}{2}
Compute the size of the giant component in a network nodes in the following networks:

\subquestion An ER-graph with $z=2.5$ with 100.000 nodes.

\solution
We know that the giant component size in an ER network obeys
\eq{
s = 1-\exp{-sz}
}
For $z=2.5$ we solve this by iteration, which yields $s\approx 0.89264$ as the fraction of nodes in the giant component. To get the number we multiply by $N$ which yields $89,264$ nodes. 

[There is an 11\% chance that a randomly picked node is not in the giant component. All other components are small in comparison, containing only a few nodes. Hence the expected number of nodes that we find in the component of a randomly picked nodes is  
approximately $79.000$ but this wasn't the question.] 

\subquestion A network consisting of 50.000 nodes of degree 1 and 50.000 nodes of degree 3.

\solution
The degree distribution is 
\eq{p_k= \frac12 (\delta_{k,1} + \delta_{k,3})}

We compute the mean degree
\eq{
z=\sum k p_k = \frac{1}{2}(1+3) = 2
}

This leads to the excess degree distribution
\eq{
q_k = \frac{(k+1)p_{k+1}}{z} = \frac{1}{4} (\delta_{k,0} + 3 \delta_{k,2})
}
Let us also compute the mean excess degree, we don't actually need it here but it will be useful in a later exercise
\eq{
q=\sum k q_k = \sum k \frac{1}{4} (\delta_{k,0} + 3 \delta_{k,2}) = 
 \frac{0 + 6}{4} = 1.5
}
To compute the probability that a link does not lead to the giant component, we use the self-consistency equation
\eqa{
v&=&\sum q_k v^k  \\
 &=&\sum \frac{1}{4} (\delta_{k,0} + 3 \delta_{k,2}) v^k \\
 &=& \frac{1 + 3 v^2}{4}
}
We can write this as quadratic polynomial
\eq{
0 = 3 v^2 - 4v +1.  
}
We could now solve this in the usual way (e.g.~completing the square). A more insightful solution is to realize that we know one solution already as $v=1$ is always a solution due to the nature of the self-consistency approach. We are not interested in this solution itself but we can divide the corresponding factor $(v-1)$ out of the polynomial, using polynomial long division:
\eq{
(3v^2-4v+1)/(v-1)=3v-1 
}
and hence the solution that we are actually interested in is 
\eq{
v=\frac{1}{3}
}
[There will actually a longer exercise on the polynomial long dividion in Chap.~10]

Now that we know $v$ we use the equation for the giant component size 
\eqa{
s&=&1-\sum p_k v^k \\
 &=&1-\sum \frac12 (\delta_{k,1} + \delta_{k,3}) v^k \\
 &=&1- \frac{v+v^3}{2} \\
 &=&1- \frac{1/3+1/27}{2} \\
 &=& \frac{54}{54} - \frac{9}{54} - \frac{1}{54}\\
 &=&  \frac{22}{27} = 0.\xoverline{814} 
}
In other words we expect 81.481 nodes to be in the giant component. 

