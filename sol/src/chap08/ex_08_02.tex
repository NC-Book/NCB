
\exercise{Abstract component sizes\label{exCowsAbstractComp}}{2}
Compute the expected size in of the component of a randomly picked node in an ER network with

\subquestion $z=0.1$

\solution 
For (a) and (b) use the formula for the size of small components
\eq{
S=1+\frac{z}{1-q}
}
Because the network is an ER-graph, $z=q$. And hence in the case (a) we get 
\eq{
S=1+\frac{0.1}{0.9}\approx 1.1  
}
This makes sense, every component contains at least the randomly picked node. If the mean degree is 0.1, then we expect less than every 10th node to have a degree of 1. So only about every tenth randomly picked node will be in a component of size 2. The chances of picking a node that is in a component containing more than two nodes are very small. 

\subquestion $z=0.9$

\solution
Now $z=q=0.9$. Substituting into the same equation as in (a) yields 
\eq{
S=1+\frac{0.1}{0.9}\approx 1  
}
and in the case (b)
\eq{
S=1+\frac{0.9}{0.1}= 10
}
As the mean degree approaches 1 the components are becoming larger quickly.
