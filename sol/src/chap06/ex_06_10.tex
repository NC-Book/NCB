\exercise{Diameter of heterogeneous networks \label{exFriendsDiameter}}{4}
Revisit our estimate of the network diameter (average node distance) from the previous chapter. In the derivation we assumed that our initial node has degree $z$. Further we assumed that if follow the links form this node we reach nodes that have in average $z$ links in addition to the one we are arriving on. This means that we assumed $q=z$. Let us now relax this assumption. Follow the reasoning of Chap.~5 to derive a formula for the diameter of a locally treelike network with a given number of nodes $N$, mean degree $z$, and mean excess degree $q$, but clustering coefficient $c=0$. Then, use your formula to estimate the diameter of a network with 10000 nodes, of which half have degree 2 and half have degree 8. 

\solution
Following the same reasoning as in the previous chapter we first estimate $n_d$ the expected number of nodes at distance $d$ from a starting node for different values of $d$.
\eqa{
n_0 &=& 1 \\
n_1 &=& z \\
n_2 &=& zq \\
n_3 &=& zq^2 
}
We start with one person at distance 0, who has $z$ neighbors. These neighbors have $q$ additional neighbors, who in turn have $q$ additional neighbors. Continuing this logic we find
\eq{
n_d = zq^{d-1}
}
for all $d>1$. To find the diameter $D$ we ask at what distance we reach practically everybody. Which leads to the condition
\eq{
N=zq^{D-1}
}
We can solve this equation for $D$ by dividing by $z$ and applying a logarithm:
\eqa{
N/z&=&q^{D-1} \\
\log_{q}(N/z)&=&D-1 \\
D&=&1+\log_{q}(N/z)
}
If we want we can use the logarithms rules to rewrite this as 
\eq{
D=1+\log_{q}(N)-\log_{q}(z) = 1 + \frac{\ln(N)-\ln(z)}{\ln{q}}.
}

The example network from the question has $N=10,000$ and 
\eq{
p_k+0.5 \delta_{k,2} + 0.5 \delta_{k,8}.
}
We compute
\eq{
z=\sum k p_k = 0.5\cdot 2 + 0.5 \cdot 8 = 5
}
and construct the excess degree distribution
\eq{
p_k = (k+1)p_{k+1}/z = (\delta_{k,1}+4\delta_{k,7})/5 = 0.2 \delta_{k,1} + 0.8 \delta_{k,7}
}
and the mean excess degree
\eq{
q = \sum k q_k = 0.2 + 7 \cdot 0.8 = 5.8 . 
}
Substituting into our diameter formula we find 
\eq{
D=1+\log_{5.8}(10000/5) = 1+\ln(2,000)/\ln(5.8) \approx 5.32, 
}
which answers the second part of the question. 

As a little bonus let's compare this to our previous network, which did not take the heterogeneity of the degree distribution into account. In the previous chapter we already derived 
\eq{
D=\log_z(N)
}
substituting yields 
\eq{
D=\log_5(10000)\approx 5.77 
}
So by taking the heterogeneity into account, we discovered that the network is actually a little bit smaller than we would have expected if we neglected the heterogeneity. In this example the difference is small because the heterogeneity is mild ($q\approx z$). By contrast, very strongly heterogeneous networks can have very small diameters, we will see this again in a later \vs{chapter}{lecture}. 