
\exercise{Degree Correlations \label{exFriendsDegreeCorr}}{4}
In a sufficiently random network the expected degree of a random neighbor of a random node is $k_{\rm nn}=q+1$. However, this isn't true if the nodes are connected in a specific way that causes strong assortativity/disassortativity.
\subquestion
For illustration, compute the average degree of a random neighbor of a random node, $k_{\rm nn}$, for the star network from \vs{Fig.~\ref{figFriendsTwoExamples}B.}{the following network: 
\begin{center}
\includegraphics[width=0.3\textwidth]{star}    
\end{center}
}
Compare $k_{\rm nn}$ to $q$.

\solution
If we pick a random node in the network then we get the hub node with probability $1/10$. If we then pick a random neighbor we are guaranteed to get a neighbor with degree 1. With probability $9/10$ we pick one of the peripheral nodes, if we then pick a random neighbor this neighbor will always be the hub, so we will end up with a node of degree $9$. So the expected degree of a random neighbor of a random node is 
\eq{
k_{\rm nn} = \frac{1}{10} 1 + \frac{9}{10}9 = 8.2
}
Due to the strong disassortativity in this network this significantly greater than $q=4$ which we know from the \vs{chapter}{lecture}. 

\subquestion
Suppose the degree distribution of a network is $p_k$ and the probability that a neighbor of a node with degree $k$ has degree $j$ is $x_{k,j}$. Find a general formula for $k_{\rm nn}$. 

\solution
To arrive at the general formula, let's first compute the expected degree of the neighbors of a node of degree $k$. We can write this as 
\eq{
\sum_j j x_{k,j}. 
}
Now if we randomly pick a node we end up picking a node with degree $k$ with probability $p_k$. Summing over all possibilities we find 
\eq{
k_{\rm nn} = \sum_k p_k \sum_j j x_{k,j}. 
}

\subquestion
Write $p_k$ and $x_{k,j}$ for network B in terms of Kronecker deltas and show that your formula for $k_{\rm nn}$ yields the expected result. 

\solution
We already know that the degree distribution of the network can be written as 
\eq{
p_k = \frac{1}{10} \delta_{k,9} +\frac{9}{10} \delta_{k,1}
}
To also find an expression for $x_{k,j}$, we can see from the network that the neighbors of all nodes of degree one have degree 9 with probability 1 
\eq{
x_{1,9} = 1.
}
Likewise, a random neighbor of a node of degree 9 has degree 1 with probability 1, i.e.
\eq{
x_{9,1} = 1.
}
All other $x_{k,j}$ are zero. (Well, actually properties such as $x_{4,5}$ are undefined as there are no nodes of degree 4 in the network, but we can treat them as zero). Hence we can write
\eq{
x_{k,j} = \delta_{k,1}\delta_{j,9}+\delta_{k,9}\delta_{j,1}
}
Have you figured this out yourself? If yes, well done, if no take a moment to substitute some pairs of numbers $(k,j)$ into this formula and to convince yourself that this is correct. 

Let's substitute $x_{k,j}$ and into our formula 
\eqa{
   k_{\rm nn} &=& \sum_k p_k \sum_j j x_{k,j} \\
     &=&  \sum_k p_k \sum_j j \left(\delta_{k,1}\delta_{j,9}+\delta_{k,9}\delta_{j,1} \right) \\
     &=& \sum_k p_k \sum_j j \delta_{k,1}\delta_{j,9}+ j \delta_{k,9}\delta_{j,1}. 
}
We use the substitution trick-vanishing act combo, first for $j$ and then for $k$, which yields
\eqa{
   k_{\rm nn} &=& \sum_k p_k \left( 9 \delta_{k,1}+ \delta_{k,9} \right)\\
     &=& \sum_k p_k \left( 9 \delta_{k,1}+ \delta_{k,9} \right)\\ 
     &=& \left(\sum_k 9p_k \delta_{k_1}\right) + \left(\sum_k p_k \delta_{k,9} \right) \\
     &=& 9p_1 + p_9.
}
Of course we could have substituted $p_k$ in already earlier, but waiting with this saved us some writing. So, the last thing to do now is to plug in $p_k$ to find
\eq{
   k_{\rm nn} = 9 \frac{9}{10} + \frac{1}{10} = 8.2,
}
which is the expected result. 
