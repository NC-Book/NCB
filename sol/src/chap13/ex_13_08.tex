\exercise{Fixed points and the logistic map}{3}
A famous map, is the so-called logistic map 
\eqn{
x_{i+1}=px_i(1-x_i)
}
where $p$ is a parameter. This is a nonlinear map, so we can't just solve it like the bee system. However, we can compute \emph{fixed points}, i.e.~points which remain stationary under the action of the map, analogous to the steady states in differential equations. 

\subquestion 
For a general map $\vec{x_{i+1}}=f(\vec{x_i})$ formulate the condition 
that a value $\vec{x^*}$ has to meet to be be considered a stationary solution (Hint: It's not $f(\vec{x^*}) = \vec{0}$.)

\solution
We want the point not to change when the map is applied, so 
\eqn{f(\vec{x^*}) = \vec{x^*}.}

\subquestion
Compute the fixed points of the logistic map. Verify your result by showing that if you substitute the fixed points into the map you get the same result back. 

\solution
We need to solve 
\eqn{
x^* = p x^* (1-x^*) 
}
Since every term contains a factor of $x^*$, one solution is 
\eqn{
x_1^* = 0. 
}
To search for other solutions we divide the equation by $x^*$, which yields
\eqan{
1 &=& p (1-x^*) \\
1/p &=& 1-x^* \\
1/p-1 &=& -x^* \\
(p-1)/p &=& x^*
}
hence the second fixed point is 
\eqn{
x_2^* = \frac{p-1}{p}
}

We check these results by substituting them into the map. For the first fixed point we find 
\eqn{
f(x_1^*)=p\cdot 0 \cdot (1-0) = 0 = x_1^*  
}
so we have shown that $x_1^*=0$ remains stationary under the action of the map. It is indeed a fixed point. 

Similarly, for the second fixed point we check
\eqan{
f(x_2^*)&=&p\frac{p-1}{p}\left(1-\frac{p-1}{p}\right) \\
   &=& (p-1)\left(1-\frac{p-1}{p}\right) \\
   &=& (p-1)-\frac{(p-1)^2}{p} \\
   &=& \frac{p(p-1)-(p-1)^2}{p} \\
   &=& \frac{(p-1)(p-(p-1))}{p} \\
   &=& \frac{(p-1)}{p}(p-p+1)) \\
   &=& \frac{(p-1)}{p} = x_2^* 
}
which confirms that also $x_2^*$ is indeed a fixed point.

