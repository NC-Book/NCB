\exercise{Going in circles, again}{3}
In Ex.~12.10 we studied the following system using a change of variables to disentangle the differential equations:
\eqan{
\dot{x} &=& y  \\ 
\dot{y} &=& -x
}
The previous solution was elegant but required a lot of insight. Now, solve the system also using the eigendecomposition method from this chapter. Also find the particular solution starting at $x=1$, $y=0$. (There may be a surprise on the way but eventually you should reach the same result as before. If it is too difficult now read the next chapter first, and then perhaps check the solution.) 

\solution

We write the system in the matrix form 
\eq{
\underbrace{\avec{\dot{x} \\ \dot{y} }}_{\vec{\dot{x}}} = \underbrace{\avecc{0 & 1 \\ -1 & 0}}_{\bf J} \underbrace{\avec{x\\ y}}_{\vec{x}}
}
Then we compute the eigenvectors and eigenvalues of $\bf J$ using 
\eq{
\avecc{0 & 1 \\ -1 & 0}\avec{1 \\b } = \lambda \avec{1 \\b }
}
which can be written in components as 
\eqa{
b &=& \lambda  \\
-1 &=& b\lambda 
}
Substituting the first equation into the second yields 
\eq{
-1 = \lambda^2
}
So $\lambda$ now has the imaginary values 
\eq{
\lambda_1 = i \qqq \lambda_2 = -i
}
where $i=\sqrt{-1}$. Since $b=\lambda$ the corresponding eigenvectors are
\eq{
\vec{v_1} = \avec{1 \\ i} \qqq \vec{v_2} = \avec{1 \\ -i}  
}
Hence the general solution is 
\eq{
\vec{x}(t) = c_1 \avec{1 \\ i} \exp{it} + c_2 \avec{1 \\ -i}  \exp{-it}.
}
We can use the initial condition from the exercise ($x=1$, $y=0$) to determine the value of $c_1$, $c_2$ For this purpose we consider the system at $t=0$, which yields 
\eq{
\avec{1\\ 0} = c_1 \avec{1 \\ i} + c_2 \avec{1 \\ -i} 
}
We can now immediately see that we need $c_1 = C_2$ to satisfy the condition in the lower row. And then using $c_1=c_2$ the upper row gives us $1=2c_1$, so the coefficients are 
\eq{
c_1 =c_2 = \frac{1}{2}
}
So now our full solution reads 
\eq{
\vec{x}(t) = \frac{1}{2} \avec{1 \\ i} \exp{it} + \frac{1}{2} \avec{1 \\ -i}  \exp{-it},
}
but we can make it even nicer. Note that the first component is 
\eq{
x(t) = \frac{\exp{it}+\exp{-it}}{2} = \cos(t)
}
and the second component is 
\eq{
y(t) = \frac{\exp{it}-\exp{-it}}{2i} = \sin(t)
}
and hence we find the expected solution 
\eqa{
x(t) &=& \cos(t) \\
y(t) &=& \sin(t)
}





