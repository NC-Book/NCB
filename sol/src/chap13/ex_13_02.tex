

\exercise{Eigenvalues and eigenvectors}{1}
\label{exBeeEV}%
Compute the eigenvalues and eigenvectors of the matrices  
\eqn{
{\bf A}=\avecc{6 & 5 \\ 12 & 2}\qqq {\bf B}=\avecc{1 & 8 \\ 0 & 3 } }
Then also write the vector $(8,9)^{\rm T}$ as a linear combination of the eigenvectors of matrix $\bf A$. 

\solution 
For ${\bf A}$ we need to solve 
\eq{
\avecc{6 & 5 \\ 12 & 2}\avec{a \\ b} = \lambda \avec{a \\ b} 
}
Unless $a=0$ we can normalize the vector such that $a=1$. Hence we need to solve 
\eq{
\avecc{6 & 5 \\ 12 & 2}\avec{1 \\ b} = \lambda \avec{1 \\ b} 
}
or in components
\eqa{
6+5b &=& \lambda \\
12+2b &=& \lambda b 
}
Solving the first equation for $b$ yields 
\eq{
\label{eqv2}
b = \frac{\lambda-6}{5},
}
and substituting this back into the second equation gives us the condition 
\eq{
12 + 2 (\lambda-6)/5 = \lambda (\lambda-6)/5 
}
We multiply both sides by 5 to avoid the fractions and then solve
\eqa{
60 + 2 (\lambda-6) &=& \lambda (\lambda-6) \\
48  &=& \lambda^2-8\lambda  \\
48  &=& (\lambda-4)^2-16  \\
64  &=& (\lambda-4)^2  \\
\pm 8 &=& \lambda-4  \\
4 \pm 8 &=& \lambda  
}
So we have two eigenvalues $\lambda_1 = 12$ and $\lambda_2=-4$. Using Eq.~(\ref{eqv2}) we compute the corresponding values of $b$ which are 
$6/5$ for the first eigenvector and $-4/5$ for the second one. To avoid the fractions we can multiply the eigenvectors by a factor of five which yields 
\eq{
\vec{v_1} = \avec{5\\ 6} \qqq \vec{v_2} = \avec{5\\ -4}
}

To decompose the vector $(8,9)^{\rm T}$ into these eigenvectors we need to solve 
\eq{
  \avec{8\\ 9} = c_1 \avec{5\\ 6} + c_2 \avec{5\\ -4}
}
and hence 
\eqa{
8 &=& 5 (c_1+c_2) \\
9 &=& 6c_1 -4c_2 
}
We can rewerite the first equation as 
\eq{
c_1+c_2 = 8/5
}
or 
\eq{
c_2=8/5-c_1.
}
Substituting into the second equation yields
\eqa{
9 &=& 6 c_1 - 4 ( \frac85 - c_1) \\
9 &=& 10 c_1 - \frac{32}{5}  \\
77/5 &=& 10 c_1   \\
c_1 &=& 77/50
}
and correspodingly 
\eq{
c_2 = \frac{80}{50}- \frac{77}{50} = \frac{3}{50}. 
}

Now we also want to know the eigenvalues and eigenvectors of the matrix $\bf B$ but this is much less tedious. We consider
\eq{
\avecc{1 & 8 \\ 0 & 3 }\avec{a\\ b} = \lambda \avec{a\\ b}
}
This time the second equation gives us one solution for $\lambda$ straight away. It reads, 
\eq{
\label{eigencond}
3b = \lambda b 
}
So $\lambda=3$ is an eigenvalue. To find the corresponding eigenvectors we consider  
\eq{
\avecc{1 & 8 \\ 0 & 3 }\avec{a \\ b} = 3 \avec{a\\ b}
}
The second row is now automatically solved, while the first row yields the condition 
\eq{
a +8 b = 3 a 
}
For convenience we can normalize the vectors such that $b=1$. In this case 
\eq{
a+ 8 = 3a 
}
so an $8=2a$ and hence the vector is $(4,1)^{\rm T}$. 

There should also be a second eigenvector, but the only other way to solve Eq.~(\ref{eigencond}) is $b=0$, so let's try this:
\eq{
\avecc{1 & 8 \\ 0 & 3 }\avec{a \\ 0} = \lambda \avec{a\\ 0}.
}
Again we have already solved the second row, but the condition from the first row now reads
\eq{
a = \lambda a
}
Clearly, $\lambda=1$ and then we can chose $a$ however we like, so we might as well use $a=1$. So in summary the matrix ${\bf B}$ hast the eigenvectors 
\eq{
\vec{v_1} = \avec{4\\ 1} \qqq \vec{v_2}=\avec{1\\ 0}  
}
and the corresponding eigenvalues are $\lambda_1 = 3$ and $\lambda_2=1$.

So the eigenvalues are also the values that appear on the diagonal of the matrix. Such diagonal values are always eigenvalues when the matrix is triangular, that is if all elements either above or below the diagonal are zero. 

