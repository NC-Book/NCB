
\exercise{Abstract two-dimensional map}{2}
Consider the following system: 
\eqan{
X_{i+1} &=& -X_i + Y_i \\ 
Y_{i+1} &=& X_i-Y_i
}
\subquestion Write the system in matrix form.

\solution
We write
\eqn{
\vec{x_{i+1}} = {\bf J} \vec{x_i} 
}
where 
\eqn{
{\bf J}=\avecc{-1 & 1 \\ 1 & -1}
}
and $\vec{x_i}=(X_i,Y_i)^{\rm T}$. 

\subquestion
Compute the eigenvalues of the matrix that appears in (a).

\solution
We write the characteristic polynomial 
\eqan{
0&=&(-\lambda-1)(-\lambda-1)-1 \\
0&=&(\lambda+1)(\lambda+1)-1 \\
0&=&\lambda^2+2\lambda 
}
At this point we can see that $\lambda_1=0$ and $\lambda_2=-2$. 

\subquestion 
Find the corresponding eigenvectors and write the initial state $X_0=2$, $Y_0=0$ as a linear combination of eigenvectors. 

\solution
For the first eigenvector we consider the first line of the matrix which leads 
to the condition 
\eqn{
 \avecc{-1 & 1 \\ 1 & -1} \avec{X\\ Y} = 0 \avec{X\\ Y} 
}
considering either line leads to 
\eqn{
X=Y
}
and hence eigenvectors the eigenvector
\eqn{
\vec{v_1} = \avec{1\\1}
}
The second eigenvector must obey the condition
\eqn{
 \avecc{-1 & 1 \\ 1 & -1} \avec{X\\ Y} = -2 \avec{X\\ Y} 
}
Considering the first line leads to 
\eqn{
-X+Y=-2X
}
or in other words $Y=-X$ and hence to
\eqn{
\vec{v_2}=\avec{1\\ -1}
}
or multiples thereof. With these eigenvectors it is straight fornward to expand the initial state as
\eqn{
\avec{2 \\ 0} = \avec{1 \\ 1} + \avec{1 \\ -1}
}
(If you don't see this immediately, try the ansatz $\vec{x_0}=c_1 \vec{v_1}+c_2 \vec{v_2}$ then consider the lines separately. They will say $c_2=c_1$ and $c_1+c_2=2$ respectively which leads to $c_1=c_2=1$.)

\subquestion
Hence, solve the initial value problem (i.e.~find $X_i$, $Y_i$ for all $i>0$).

\solution
For $i>0$ we can write 
\eqan{
\vec{x_i} &=& {\rm J}^i \vec{x_0} \\ 
    &=& {\rm J}^i (\vec{v_1}+\vec{v_2}) \\
    &=& {\rm J}^i\vec{v_1}+{\rm J}^i\vec{v_2} \\
    &=& {\lambda_1}^i\vec{v_1}+{\lambda_2}^i\vec{v_2} \\
    &=& (-2)^i \vec{v_2} \\
}
Hence, the solution is 
\eqn{
X_i = (-2)^i  \hspace{1cm} Y_i = -(-2)^i 
}
\subquestion 
Verify your solution by computing the first $X_1$, $Y_1$, $X_2$, $Y_2$ by hand.

\solution
Using the formula from the question
\eqn{
X_1 = -X_0 + Y_0 = -2 = (-2)^1
}
\eqn{
Y_1 = X_0 - Y_0 = 2 = -(-2)^1
}
\eqn{
X_2 = -X_1 + Y_1 = 2+2 = 4 = (-2)^2 
}
\eqn{
Y_2 = X_1 - Y_1 = -2-2 = -4 = -(-2)^2 
}
So, this works. 

\subquestion
Bonus: Using matrix methods, show that for any initial condition $X_i=-Y_i$ must hold for all $i\geq 1$.

\solution
For a given initial state $\vec{x_0}$ we can find $c_1$ and $c_2$ such that
\eq{
\vec{x_0} = c_1 \vec{v_1} + c_2 \vec{v_2}
}
Starting from this state the general solution is 
\eq{
\vec{x_i} = {\bf J}^i \vec{x_0} = c_1 {\lambda_1}^i \vec{v_1} + c_2 {\lambda_2}^i\vec{v_2}
}
Because $\lambda_1=0$ this simplifies to 
\eq{
\vec{x_i} = c_2 {\lambda_2}^i\vec{v_2}
}
which means
\eqa{
X_i &=& c_2 (-2)^i \\
Y_i &=& -c_2 (-2)^i
}
And hence $X_i=-Y_i$.
