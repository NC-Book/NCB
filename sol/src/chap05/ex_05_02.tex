
\exercise{Logarithm revision\label{exSWlog}}{1}
Note that $\log_{10}(2)\approx 0.3$. Use this to solve the following equations without a calculator: a) $20=10^x$, b) $4000=10^x$ c) $2=1000^x$ d) $5=10^x$ e) $2^x=10^9$. If you can't do it, check out the explanations in the solutions immediately.

\solution
The logarithm $x=\log_a{y}$ is (by definition) the solution to \eq{y=a^x}. Some useful consequences of this definition are 
\eqa{
\mbox{i)} &\quad& a^{\log_a(x) } = x = \log_a(a^x) \\
\mbox{ii)} &\quad& \log_a(xy) = \log_a(x) + \log_a(y) \\
\mbox{iii)} &\quad& \log_a(x^y) = y\log_a(x)  \\
\mbox{iv)} &\quad& \log_a(x) = \log_b(x) / \log_b(a) \\ 
\mbox{v)} &\quad& \log_a(1/x) = -\log_a(x) \\
\mbox{vi)} &\quad& \log_x(x) = 1
}

\noindent{}To solve (a) we use
\eq{
x=\log_{10}(20) = \log_{10}(2)+\log_{10}(10) \approx 0.3 + 1 = 1.3
}
Note that the second term, where we used rule ii, and then rule vi.\\[0.5em] 
For part (b) we can do the following:
\eq{
x=\log_{10}(8000)=\log_{10}(8)+\log_{10}(1000) = \log_{10}(2^3)+\log_{10}(10^3)\approx 3\cdot 0.3+3 \approx 3.9 
}
where we used rule ii, then wrote the numbers as powers to apply rule iii.\\[0.5em]
To solve (c), we need rule iv
\eq{
x=\log_{1000}(2)=\frac{\log_{10}(2)}{\log_{10}(1000)}\approx \frac{0.3}{3}=0.1 
}\\[0.5em]
We do part (d) almost in the same way
\eq{
x=\log_{10}(5)= \log_{10}(10/2) = \log_{10}(10)-\log_{10}(2)\approx 1-0.3 = 0.7
}
The key idea here is to write 5 as 10/2 and then use rules ii and v.
\\[0.5em]
Finally for (e) we start with a change of basis (rule iv)
\eq{
x=\log_{2}(10^9)= \frac{\log_{10}(10^9)}{\log_{10}(2)}\approx \frac{9}{0.3} = \frac{90}{3} = 30
}
