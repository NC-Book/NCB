
\exercise{A real problem\label{exSWlabor}}{3}
In most western countries child labor has been illegal since the worker's rights movement in the beginning of the 20th century. Some countries also make it illegal for companies to do business with suppliers that use child labor. Suppose a new law is passed that also makes it illegal to have suppliers who have suppliers that use child labor. You know that your company does not use child labor, nor does any of your suppliers. But you are not sure about their suppliers. Assuming that there are $10^6$ companies and 1000 of these use child labor. A typical company has about 600 suppliers. Given these numbers, do we have to worry that a supplier of your suppliers uses child labor?

\solution
As we don't know the clustering coefficient (and it is probably relatively low) our estimate is that the number of our supplier's suppliers could be a s high as 36.000. As one company in 1.000 uses child labor, we would expect that 36 of our supplier's suppliers use child labor. 

This is an important real world problem. For a firm, mapping all of their supplier's suppliers is very hard as it requires the supplier's cooperation, however not knowing who your supplier's suppliers are can carry significant legal, reputational and business disruption risks. The network of business relationships turns out to be a very small world, which means that disturbances anywhere in the world can disrupt production here. Their is an increasing demand for methods from complex systems theory to manage and regulate this highly complex network. 

The bigger picture here is that the global supply network has such a small diameter, that a disturbance anywhere is felt everywhere. 
