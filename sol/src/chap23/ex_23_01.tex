
\exercise{Circular reasoning}{2}
While I worked in Davis I was struggling to buy enough drinking water. Basically I distinguish three states: a) ``I have enough water'',  b) ``Water in short supply'', c) ``Not a single drop''. When I am in state a, I usually transition to state b at rate 1. If I am in state b, then I try to buy water. This happens happens also at rate 1 and takes me back to state a. More commonly, at rate 2, I transition to state c. In state c I have a good incentive to go shopping. This happens at rate 3 and takes me back to state a. However, sometimes when I am in state c, I just grab one bottle on the way back from work. This takes me back to state b at rate 1.  Use Kirchhoff's theorem to determine the proportion of the time I spend in state c. 

\solution

The setting from the question is described by the Adjacency matrix 
\eq{
{\bf A} = \aveccc{0 & 1 & 3 \\ 1 & 0 & 1 \\ 0 & 2 & 0}
}
this leads to the Laplacian
\eq{
{\bf L} = \aveccc{1 & -1 & -3 \\ -1 & 3 & -1 \\ 0 & -2 & 4}
}
We now compute the determinants of the minors. For node 1 this is 
\eq{
m_1=\left| \begin{array}{cc}  3 & - 1 \\ -2 & 4  \end{array} \right| = 12-2=10
}
For node 2 
\eq{
m_2=\left| \begin{array}{cc}  1 & - 3 \\ 0 & 4  \end{array} \right| = 4-0=4
}
And, for node 3
\eq{
m_3=\left| \begin{array}{cc}  1 & -1 \\ -1 & 3  \end{array} \right| = 3-1=2
}
So the total proportion of time I spend without water is 
\eq{
T_3= \frac{m_3}{m_1+m_2+m_3} = \frac{2}{2+4+10} = \frac{1}{8}
}
On the other hand I spend half the time with plenty of water. 

