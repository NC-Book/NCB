\exercise{Solving SIS and Logistic Growth}{4}
\label{exGrowSIS}
Many systems in nature exhibit logistic growth, which is described by the logistic differential equation  $\dot{x}=Ax(K-x)$.

\subquestion Show that the equation for the SIS model can be written in the form of logistic growth and determine the values of $A$ and $K$.

\solution 
In the logistic differential equation the variable is called $x$ while in the SIS model we called it $I$, so lets start by renaming, $I=x$. In terms of $x$ we can write our SIS model as 
\eq{
\dot{x}=p(1-x)x -rx
}
To bring it into the form of the logistic differential equation it is usefull to first collect all powers of $x$,
\eq{
\dot{x} = (p-r)x - p x^2  
}
Now we can isolate one factor of $x$, which gives us
\eq{
\dot{x} = x\left((p-r)- p x\right).  
}
We have arrived almost at the desired form however, there is still a factor $p$ in front of the $x$ in the bracket, whereas there is only a 1 there in the logistic growth. Pulling this p out yields
\eq{
\dot{x} = xp\left(\frac{(p-r)}{p}- x\right).  
}
The equation now has the right form. We can identify 
\eq{
K=\frac{p-r}{p}\qqq  A=p
}
which yields
\eq{
\dot{x}=Ax(K-x).
}

\subquestion
Show that the logistic differential equation can be solved by separation of variables, identify the two integrals that need to be solved, and solve the simpler one of them.  

\solution 
We can write the equation in the variational form 
\eq{
\frac{1}{x(K-x)} {\rm d}x = A {\rm d} t 
}
This leads us to the integral equation 
\eq{
\int \frac{1}{x(K-x)} {\rm d}x = \int A {\rm d} t 
}
The integral on the right is easy and we can solve it straight away
\eq{
\int \frac{1}{x(K-x)} {\rm d}x = At+C 
}

\subquestion
We are now left with a more difficult integral to solve. To do this, first show that the term under the integral can be written in the form
\eqn{
 \frac{a}{x}+\frac{b}{K-x}  
}
where $a$ and $b$ are constants. In other words: Find $a$ and $b$ such that such that the expression above becomes identical to the term under the integral. (Hint: After the first steps, isolate the $x$.)

\solution 
Consider that the term under our integral was 
\eqn{
 \frac{1}{x(K-x)}.
}
We want to show that it can be written in the form 
\eqn{
 \frac{a}{x}+\frac{b}{K-x}.  
}
Hence we demand 
\eq{
 \frac{1}{x(K-x)} = \frac{a}{x}+\frac{b}{K-x}.
}
To solve this we multiply everything with $x(K-x)$ and take advantage of the resulting cancellations, which yields 
\eq{
1 = (K-x)a + xb
}
which we can also write as 
\eq{
0 = (Ka-1) + (b-a)x
}
This condition must be fulfilled for all values of $x$ which is only possible if we chose $a=b$. From the first bracket we see furthermore that $a=1/K$.

Let's do a quick check to see if this is right. 
\eq{
\frac{1}{K(K-x)}+\frac{1}{Kx}=\frac{x}{K(K-x)x}+\frac{(K-x)}{K(K-x)x}=\frac{x+K-x}{K(K-x)x}=\frac{1}{x(K-x)}
}
OK, this works. This trick of splitting the fraction into multiple simpler fractions is called ``continued fractions'' by the way. It appears in the solution of many difficult problems. 

\subquestion
Use the result from (c) to split the integral that we still need to solve into two integrals and solve them. 

\solution 
We rewrite the integral
\eq{
\int \frac{1}{x(K-x)} {\rm d}x = \int \left( \frac{1}{K(K-x)} + \frac{1}{Kx}\right) {\rm d} x
}
and then split it 
\eq{
\int \left( \frac{1}{K(K-x)} + \frac{1}{Kx}\right) {\rm d} x 
= \int \frac{1}{K(K-x)} {\rm d}x + \int \frac{1}{Kx} {\rm d} x
}
This gives us two simpler integrals to solve. We can solve the second one straight away
\eq{
\int \frac{1}{Kx} {\rm d} x = \frac{1}{K} \int \frac{1}{x} {\rm d} x = \frac{1}{K} \log(x) 
}
Logarithms frequently show up when we integrate fractions. To solve the first of the two integrals, let's just try what happens if we differentiate $log(K-x)$, using the chain rule 
\eq{
\frac{{\rm d}}{{\rm d}x} \log(K-x) = \frac{K-x}{-1} 
}
where the $-1$ appears due to the inner derivative. If multiply the whole equation and multiply by -1 and then integrate it again we get 
\eq{
-\log(K-x) = \int \frac{1}{K-x} {\rm d}x 
}
and hence 
\eq{
\int \frac{1}{K(K-x)} {\rm d}x = - \frac{\log(K-x)}{K}
}
\subquestion Finally put all the parts together and solve for $x$. 
\solution
In summary: We started from 
\eq{
\dot{x}=Ax(K-x)
}
which leads us to the integral equation 
\eq{
\int \frac{1}{x(K-x)} {\rm d}x = \int A {\rm d} t 
}
We can rewrite the left hand side as two integrals 
\eq{
\frac{1}{K} \int \frac{1}{x} {\rm d} x + \frac{1}{K} \int \frac{1}{K-x} {\rm d}x   = \int A {\rm d} t 
}
and then solve all the integrals
\eq{
\frac{1}{K} \log(x) - \frac{1}{K} \log(K-x) = At+C
}
We now need to solve for $x$ to do we fist multiply everything by $K$,
\eq{
\log(x) - \log(K-x) = K(At+C) 
}
and then apply the exponential function to both sides
\eq{
\exp{\log(x)-\log(K-x)} = \exp{K(At+C)}.
}
We simplify the left-hand side of this equation 
\eq{
\exp{\log(x)-\log(K-x)}=\frac{\exp{\log(x)}}{\exp{\log(K-x)}} = \frac{x}{K-x} 
}
so the equation reads now 
\eq{
\frac{x}{K-x} = \exp{K(At+C)}.
}
We can now multiply both sides by $K-x$, 
\eq{
x = (K-x) \exp{K(At+C)},
}
and gather all $x$ on the left side, 
\eq{
x(1+\exp{K(At+C)}) = K \exp{K(At+C)}, 
}
to arrive at the solution 
\eq{
x = \frac{K \exp{K(At+C)}}{1+\exp{K(At+C)}}
}
At $t=0$ the solution reads 
\eq{
x_0 = \frac{K\exp{KC}}{1+\exp{KC}} 
}
we can use this to write 
\eqa{
x_0(1+\exp{KC}) &=& K \exp{KC} \\
x_0 &=& \exp{KC}(K-x_0)   \\
\exp{KC} = \frac{x_0}{K-x_0}
}
we can use this to write the solution as 
\eq{
x(t) = \frac{K \frac{x_0}{K-x_0} \exp{KAt}}{1+\frac{x_0}{K-x_0}\exp{KAt}}
}
... and simplifying a little bit 
\eq{
x(t) = \frac{Kx_0\exp{KAt}}{K+x_0(\exp{KAt}-1)}
}
This answers the question. 

Bonus: To understand the dynamics, consider that at $t=0$ we have $x=x_0$. Initially, that is, the exponential terms are very close to one. This means we can approximate 
\eq{
\exp{KAt}-1 \approx 0
}
Hence the bracket in the denominator vanishes and the dynamics looks like 
\eq{
x(t) = x_0\exp{KAt} \quad\quad\mbox{for $t\ll 1$} 
}
At later times the exponential term becomes much larger than both the $-1$ and the $K$ in the denominator, so the dynamics looks like
\eq{
x(t) = \frac{Kx_0\exp{KAt}}{x_0\exp{KAt}} = K \quad\quad\mbox{for $t\gg 1$} 
}
So if we wait long enough the growth saturates in the steady state at $x=K$.

\subquestion{Substitute $K$ and $A$ into the solution to find a general solution for the SIS model.}

\solution
Writing the solution in terms of $I$ we have 
\eq{
I(t) = \frac{KI_0\exp{KAt}}{K+I_0(\exp{KAt}-1)}
}
We now recall 
\eq{
K=\frac{p-r}{p}\qqq  A=p
}
It is nice that the solution contains the factors $KA$ which simplify to 
\eq{
KA=p-r
}
Substituting we can write our solution for the SIS model as
\eq{
I(t) = \frac{(p-r)I_0\exp{(p-r)t}}{(p-r)+I_0p(\exp{(p-r)t}-1)}
}
where we have multiplied the numerator and denominator of the fraction by $p$ to avoid a fraction inside the fraction. 
