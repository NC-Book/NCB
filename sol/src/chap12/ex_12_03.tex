\exercise{Infinite excess degree \label{exGrowExcess}}{1}
Consider a network with $p_0=0$ and $p_k=a k^{-3}$ for $k>0$. Show that this network has finite mean degree but infinite mean excess degree. You can use the $\sum_{k=1}^\infty k^{-2}$ is finite while $\sum_{k=1}^\infty k^{-1}=\infty$. [Hint: Don't try to determine $a$, it's complicated.]

\solution
The mean degree of our network is 
\eqa{
z&=&\sum k p_k \\
 &=&\sum_{k=1}^\infty k a k^{-3} \\
 &=& a \sum_{k=1}^\infty a k^{-2} 
}
which is finite. We now compute the excess degree distribution 
\eqa{
q_k &=& \frac{(k+1)p_{k+1}}{z} \\
    &=& \frac{a}{z} (k+1)(k+1)^{-3} = \frac{a}{z} (k+1)^{-2}.
}
Note that we can use this equation for all $k$ including 0.  
In the next step we want to show that the mean excess degree is infinite, so our strategy is to create an expression of the form $\sum_{k=1}^\infty k^{-1}$. We compute
\eqa{
 q&=&\sum k q_k \\
  &=& \sum \frac{a}{z} k (k+1)^{-2} \\
  &=& \sum \frac{a}{z} (k+1) (k+1)^{-2} - \frac{a}{z}(1)(k+1)^{-2}  \\
  &=& \left(\sum \frac{a}{z} (k+1) (k+1)^{-2}\right) - \left(\sum \frac{a}{z}\frac{a}{z}(k+1)^{-2}\right)  \\
  &=& \frac{a}{z} \left(\sum  (k+1)^{-1}\right) - \frac{1}{z}\left(\sum a(k+1)^{-2}\right)  \\
  &=& \frac{a}{z} \left(\sum_{k=1}^\infty  k^{-1}\right) - \frac{1}{z}\left(\sum_{k=1}^\infty ak^{-2}\right)  \\    
  &=& \frac{a}{z} \left(\sum_{k=1}^\infty  k^{-1}\right) - \frac{z}{z} \\
  &=& \frac{a}{z} \left(\sum_{k=1}^\infty  k^{-1}\right) - 1 \\
  &=& \infty - 1 = \infty \\
}