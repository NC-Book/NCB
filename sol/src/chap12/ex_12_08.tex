
\exercise{Iterative integration \label{exGrowIter}}{3}
Consider the equation
\eqn{
\dot{x}=rx
}
where $x(0)=x_0=1$. We know that we can't solve this equation by direct integration. However, let's try nevertheless ...

\subquestion
Assume that x on the right hand side of the equation is a constant and directly integrate the equation. (This will prove the assumption wrong)

\solution
So we literally assume $x=1$. substituting into the equation yields 
\eq{
\dot{x} = r
}
Integrating both sides we find 
\eq{
x(t)=1+rt.
}

\subquestion Now take your solution from (a) and use this as a new assumption, for $x$ and integrate again.

\solution 
So we now assume 
\eq{
x(t) = 1+rt
}
and hence integrate
\eq{
\dot{x}=r(1+rt) = r+r^2t, 
}
which yields
\eq{
x(t)=1+rt + \frac{r^2t^2}{2}.
}

\subquestion Iterate this process to find successively better approximations to the solution. Can you spot the pattern that is developing?

\solution
Integrating a few more times we can see that we are approaching the solution 
\eqa{
x(t)&=&1+rt + \frac{r^2t^2}{2} + \frac{r^3t^3}{6} + \ldots \\
  &=& \sum_n \frac{(rt)^n}{n!} \\
  &=& \exp{rt}
}
which is of course as it should be.
