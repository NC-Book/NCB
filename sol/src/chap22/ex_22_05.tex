
\exercise{Spectra of Hypercubes}{4}
A hypercube is a generalization of cubes to arbitrary dimensions. The 0-dimensional hypercube is just a single  node. 

\subquestion
Compute the eigenvalue of the adjacency matrix of the zero-dimenional hypercube, ${\bf A_0}$.

\solution
Since 
\eq{ {\bf A_0} = (0)}
we get $\lambda=0$

\subquestion
To make a 1-dimenional hypercube we take the zero-dimensional hypercube and make a copy of it. Then we connect every node in the original cube with the same node in the copy. Compute the eigenvalues of the corresponding adjacency matrix ${\bf A_1}$. 

\solution
Following the instructions leads to 
\eq{
{\bf{A}} = \avecc{0 & 1 \\ 1 & 0}
}
hence the eigenvalues are $\lambda_{1,2}=\pm 1$.

\subquestion
To make a 2-dimenional hypercube we take the one-dimensional hypercube and make a copy of it. Then we connect every node in the original 1D-cube with the same node in the copy. Write the corresponding adjacency matrix $\bf A_2$ as a sum  of two Kronecker products containing $\bf A_1$, the identity matrix $\bf I$, and the matrix
\eq{
{\bf M}=\avecc{0 & 1\\ 1 & 0}
}
Explain the resulting formula in words and use it to compute the eigenvalues. 
(Hint: $\bf M$ and $\bf A_1$ look the same but they fill a different role. If you used them cleverly you can shift the indices in your formula and you will get an equation that relates $\bf A_0$ and $\bf A_1$ correctly.)

\solution
The adjacency matrix for the 2D hypercube is 
\eq{
{\bf A_2} = \left(  \begin{array}{c c c c} 
0 & 1 & 1 & 0 \\
1 & 0 & 0 & 1 \\
1 & 0 & 0 & 1 \\
0 & 1 & 1 & 0 \\ 
\end{array}  \right)
}
which we can write as 
\eq{
{\bf A_2} = {\bf I} \otimes {\bf A_1} + {\bf M}\otimes {\bf I}
}
The first term of the sum represents the copying of the previous network. The second term connects nodes up with their copies.

We can test if we are on the right track by following up on the hint from the question. Which suggests that by shifting the index we can get 
\eq{
{\bf A_1} = {\bf I} \otimes {\bf A_0} + {\bf M}\otimes {\bf I}
}
We can that this is correct by substituting
\eq{
\avecc{0 & 1 \\ 1 & 0} = \avecc{1 & 0 \\ 0 & 1} \otimes (0) + \avecc{0 & 1 \\ 1 & 0}\otimes (1)
}
We can now use the Kronecker product formula to compute the eigenvalues Using the ansatz $\vec{u}\otimes\vec{v}$ for the eigenvector, we find
\eqa{
{\bf A_2}(\vec{u}\otimes \vec{v})& =& {\bf I}\vec{u} \otimes {\bf A_1}\vec{v} + {\bf M}\vec{u}\otimes {\bf I}\vec{v} \\
&=& \vec{u} \otimes \alpha_1\vec{v} + \mu\vec{u}\otimes \vec{v} \\
&=& (\alpha_1+\mu) (\vec{u} \otimes \vec{v}) 
}
Where $\alpha_1$ is any eigenvalue of $\bf A_1$ and $\mu$ is either of the two eigenvalues of $\bf M$.

Because the eigenvalues of $\bf M$ are $+1$ and $-1$ we find half the eigenvalues of $\bf A_2$ by adding to every eigenvalue of $\bf A_1$ and the other half of eigenvalues of $\bf A_2$ by substracting one from the eigenvalues of $\bf A_1$. Since the eigenvalues of $\bf A_1$ were 1 and -1, the eigenvalues of $\bf A_2$ are
\eq{
\lambda = \{ -2, 0,2 \}  
}
where the 0 appears with multiplicity 2. 

\subquestion
To make a 3-dimenional hypercube we take the two-dimensional hypercube and make a copy of it. Then we connect every node in the original 2D-cube with the same node in the copy. Compute its spectrum. 

\solution 
This is easy now. The same formula for the eigenvalues still holds. So substracting one from the eigenvalues of the two-dimensional cube yields
\eqn{
\{ -3,-1,-1,1 \}   
}
where we have written the -1 twice as an easy way to keep the multiplicity of the eigenvalue in mind. Adding one to the spectrum yields
\eqn{
\{ -1,1,1,3 \}   
}
and by putting both parts together we find the whole spectrum
\eqn{
\{-3,-1,-1,-1,1,1,1,3 \}
}

\subquestion
We can also find the Laplacian spectrum very easily. Note that the hypercubes are regular graphs, with a degree that is identical to the dimension of the cube. So for example the three-dimendional hypercube is a 3-regular graph. 
Quickly compute the spectrum of ${\bf L_3}=3{\bf I}-{\bf A_3}$.

\solution
The minus sign in front of the adjacency just flips the spectrum, because it is symmetric this leaves it unchanged actually. Adding $3{\bf I}$ to the matrix shifts all eigenvalues by 3 units to the right, so the spectrum of ${\bf L_3}$ is
\eqn{
\{0,2,2,2,4,4,4,6 \}
}

\subquestion
Now adapt our previous rule to compute the Laplacian spectrum of the 4-dimensional hypercube.

\solution
So when we move from the three dimensional cube to the four dimensional-cube in the Laplacian spectrum also the degree of the nodes increases by one, which means we get one unit more shift to the right. So instead of adding and substracting one, we get the new spectrum by adding zero and two. So the spectrum for the 3-dimensional hypercube was
\eqn{
\{0,2,2,2,4,4,4,6 \}
}
Shifted by 2 this becomes 
\eqn{
\{2,4,4,4,6,6,6,8 \}
}
Joining these two sets we get the complete ${\bf L_4}$ spectrum
\eqn{
\{0,2,2,2,2,4,4,4,4,4,4,6,6,6,6,8 \}
}

\subquestion
Now that we have spotted the pattern, consider the generating function
\eqn{
G_{i}(x) = \sum c_{i,n} x^n
}
where $c_{i,n}$ is the multiplicity of the eigenvalue $\lambda=2n$ in the Laplacian spectrum of the $i$-dimensional hypercube. Write an iteration rule 
that relates $G_i$ and $G_{i+1}$ and hence find a closed form for $G_i$

\solution
In generating functions our update rule reads 
\eq{
G_{i+1} = G_i (1+x^2)
}
and since we know $G_0=1$ (one eigenvalue $\lambda=0$) we find
\eq{
G_i = (1+x^2)^i
}
