
\exercise{Mathematical Tools\label{exBridgesSummation}}{1}
Evaluate the following expressions (If you get stuck, check out the solution, it will explain what you need to know). 

\subquestion $\sum_{i=1}^3 i$

\solution 
This represents the sum over $i$, where $i$ runs from 1 to 3. So, 
\eq{
\sum_{i=1}^3 i= 1+2+3 = 6
}

\subquestion $\sum_{i=1}^2 3i$

\solution 
This time $i$ takes only two values $1$ and $2$, but we are summing $3i$, so we have
\eq{
\sum_{i=1}^2 3i = 3 \cdot 1 + 3 \cdot 2 = 3+6=9
}

\subquestion $\sum_j A_{2,j}$, where 
\eqn{
\label{eqBridgeAdjacency}
{\bf A} = \left(\begin{array}{c c c c} 
0 & 0 & 1 & 2 \\
0 & 0 & 1 & 2 \\
1 & 1 & 0 & 1 \\
2 & 2 & 1 & 0
 \end{array}\right).
}

\solution 
Here we aren't given bounds for the summation, but from the context we can conclude that we are meant to run $j$ over all values that make sense as a column index of matrix ${\bf A}$.
As a result the expression sums over the whole second row of the matrix 
\eq{
\sum_j A_{2,j} = A_{2,1}+A_{2,2}+A_{2,3}+A_{2,4} = 0+0+1+2=3.
}

\subquestion
$\prod_{i=3}^6 i$ 
\solution
The product sign tells us to multiply factors. In this case it runs from 3 to 6 so we get 
\eq{
\prod_{i=3}^6 i=3\cdot 4 \cdot 5 \cdot 6 = 360
}

\subquestion
$\prod (A_{n,5-n}+1)$, with $\bf A$ from above.

\solution
The first complication here is that the product sign is not annotated with the name of the index, but looking at the right-hand-side we see that a variable $n$ appears which is otherwise undetermined. So $n$ must be the index of the product. 

Furthermore, we are not given bounds but we are using $n$ as index in the $4\times 4$ matrix $\bf A$ so it needs to run from 1 to 4. We can now evaluate
\eqa{
\prod (A_{n,5-n}+1) &=& \prod_{n=1}^4 (A_{n,5-n}+1) \\
  &=& (A_{1,4}+1) \cdot (A_{2,3}+1) \cdot (A_{3,2}+1) \cdot (A_{4,1}+1) \\
  &=& (2+1) \cdot (1+1) \cdot (1+1) \cdot (2+1) \\ 
  &=& 3 \cdot 2 \cdot 2 \cdot 3 \\ 
  &=& 24
}

\subquestion $3!$

\solution 
The factorial $n!$ is a shorthand notation for
\eq{
    \prod^n_{i=1} i 
}
So in the present case 
\eq{
3! = \prod^n_{i=1} i = 1\cdot 2 \cdot 3 = 6
}