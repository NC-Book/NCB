\exercise{Two abstract two-dimensional system}{2}
Let's start with the example system from the chapter
\eqan{
\dot{x}&=&y^5-x^2 \\
\dot{y}&=&1-x
}
\subquestion 
Compute the steady state

\solution
From the equation for $\dot{y}$ we obtain the condition 
\eqn{0=1-x}
and hence 
\eqn{x^*=1}
Substituting this into the condition from the equation for $\dot{x}$ yields 
\eqn{
0=y^5-1 
}
and hence 
\eqn{y=\sqrt[5]{1}=1}

\subquestion 
Compute the Jacobian in the steady state

\solution
We compute the Jacobian elements 
\eqan{
J_{11}&=&\left.\frac{\partial \dot{x}}{\partial x}\right|_* = \left. \frac{\partial}{\partial x} (y^5-x^2) \right|_* = -2x* = -2 \\
J_{12}&=&\left.\frac{\partial \dot{x}}{\partial y}\right|_* = \left. \frac{\partial}{\partial y} (y^5-x^2) \right|_* = 5{y^*}^4 = 5 \\
J_{21}&=&\left.\frac{\partial \dot{y}}{\partial x}\right|_* = \left. \frac{\partial}{\partial x} (1-x) \right|_* = -1 \\
J_{22}&=&\left.\frac{\partial \dot{y}}{\partial y}\right|_* = \left. \frac{\partial}{\partial y} (1-x) \right|_*  = 0 \\
}
and hence 
\eqn{
{\bf J} = \avecc{-2 & 5 \\ -1 & 0 }
}

\subquestion 
Compute the eigenvalues and eigenvectors of the Jacobian

\solution 
We start from the eigenvector equation 
\eqn{
\avecc{-2 & 5 \\ -1 & 0 }\avec{A\\ B} = \lambda \avec{A\\ B} 
}
where $A$ and $B$ are placeholders for the eigenvector elements. Because we can rescale eigenvectors we can set $A=1$ (unless $A$ needs to be 0). Reading the the two rows of the equation system separately gives us 
\eqan{
-2 + 5 B &=& \lambda \\
-1 &=& \lambda B
}
The second row tells us $B=-1/\lambda$, and substituting this into the first row give us 
\eqn{
-2 -\frac{5}{\lambda}= \lambda
 }
Multiplying this equation by $\lambda$ and bringing all terms over to the right side gives us the characteristic polynomial 
\eqn{
0 = \lambda^2 + 2\lambda + 5
}
which we now solve for $\lambda$. 
\eqan{
0 &=& \lambda^2 + 2\lambda + 5 \\
0 &=& (\lambda + 1)^2 + 4 \\
-4 &=& (\lambda+1)^2 \\
\lambda &=& -1 \pm \sqrt {-4} = -1\pm 2\sqrt{-1}  
}
which we can also write as 
\eqn{
\lambda_{1,2} = -1 \pm 2i
}
where $i$ is the imaginary number.

 If we also want the eigenvectors we could now find $B$ by  using $B=-1/\lambda$, but dividing by complex numbers is tedious, so lets rather use the top row from would eigenvector equation ($-2+5B=\lambda$) which we can write as 
\eqn{
B= \frac{2}{5} + \frac{1}{5} \lambda  = \frac{1}{5} \pm \frac25 i
}
Because fractions in vectors don't look nice lets scale the whole eigenvectors by a factor of 5. The result are the vectors 
\eqn{
\vec{v_1}= \avec{5 \\ 1+2i} \qqq \vec{v_1}= \avec{5 \\ 1-2i}  
}

\subquestion 
Compute the real part of the eigenvalues and decide whether the steady state is stable or not. 

\solution 
To compute the real part we discard the imaginary part of the complex eigenvalues  
\eqn{
{\rm Re}(\lambda_{1,2}) = {\rm Re}(-1 \pm \sqrt{-2}) = -1
}
So both eigenvalues have the real part -1. Because all real parts are negative the steady state is stable.  

\subquestion 
Use the same approach to find the steady state of the following system, and determine its stability:
\eqan{
\dot{x}&=&x-y+2 \\ 
\dot{y}&=&x-3y+8 
}

\solution 
From the differential equation for $x$ we get the condition 
\eqn{0=x-y+2}
and hence 
\eqn{y=x+2} 
The equation for the change of $y$ gives us the condition 
\eqn{
0=x-3y+8=x-3(x+2)+8 = x-3x-6+8 = -2x+2   
}
and hence 
\eqn{
2x=2
}
Therefore the steady state is $x^*=1$ and $y^*=3$.

To find the Jacobian we compute the derivatives 
\eqan{
J_{11} &=& \left.\frac{\partial \dot{x}}{\partial x}\right|_* = \left.\frac{\partial}{\partial x} x-y+2 \right|_* = 1 \\
J_{12} &=& \left.\frac{\partial \dot{x}}{\partial y}\right|_* = \left.\frac{\partial}{\partial y} x-y+2 \right|_* = -1 \\
J_{21} &=& \left.\frac{\partial \dot{x}}{\partial x}\right|_* = \left.\frac{\partial}{\partial x} x-3y+8 \right|_* = 1 \\
J_{22} &=& \left.\frac{\partial \dot{x}}{\partial x}\right|_* = \left.\frac{\partial}{\partial y} x-3y+8 \right|_* = -3
}
Hence the Jacobian is 
\eqn{
{\bf J}=\avecc{1 & -1 \\ 1 & -3 }.
}
We compute the characteristic polynomial 
\eqn{
\left| \avecc{1-\lambda & -1 \\ 1 & -3-\lambda }   \right|
= \lambda^2+2\lambda-2 
}
We solve for the eigenvalues 
\eqan{
0 &=& \lambda^2+2\lambda-2 \\
0 &=& (\lambda^2+2\lambda+1)-3 \\
3 &=& (\lambda+1)^2 \\
\pm \sqrt5 &=& \lambda+1 \\
}
Therefore the eigenvalues are 
\eqn{
\lambda_{1,2}=-1\pm \sqrt{3}
}
Since $\sqrt{3}$ is a real number the real part of the eigenvalue is 
\eq{
{\rm Re}(\lambda_1) = \lambda_1 = -1 + \sqrt{5}
}
Because $\sqrt{5}>1$ this eigenvalue is positive which reveals that the steady state is unstable. 
