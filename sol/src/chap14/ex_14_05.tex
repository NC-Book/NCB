
\exercise{Walk through a more complex two-dimensional analysis}{3}
\label{exComplexComplex}
Let's go step-by-step through the analysis of the following system:
\eqan{
\dot{x} &=& a+bxy \\
\dot{y} &=& \frac12 -x
}
\subquestion
Compute the steady state of the system. Then find the Jacobian in the steady state, and compute the eigenvalues:

\solution
From the second equation we see 
\eqn{
x^* = \frac12
}
Substituting into the first equation yields the condition
\eq{
0 = a +\frac{b}{2}y^* 
}
and hence 
\eq{
y^*=-2\frac{a}{b}
}

We compute the elements of the Jacobian matrix.
\eqa{
J_{11}&=&\left. \frac{\partial \dot{x}}{\partial x}\right|_* = by^* = -2a  \\
J_{12}&=&\left. \frac{\partial \dot{x}}{\partial y}\right|_* = bx^* = b/2 \\
J_{21}&=&\left. \frac{\partial \dot{y}}{\partial x}\right|_* = -1 \\
J_{22}&=&\left. \frac{\partial \dot{y}}{\partial y}\right|_* = 0 \\
}
hence
\eq{
{\bf J} = \avecc{-2a & b/2 \\ -1 & 0 }
}
We find the characteristic polynomial using the approach 
\eqn{
\avecc{-2a & b/2 \\ -1 & 0 }\avec{1\\ B} = \lambda \avec{1\\ B} 
}
which gives us the conditions 
\eqan{
-2a + bB/2  &=& \lambda \\
-1 &=& \lambda B
}
From the second condition we get $B=-1/\lambda$. Substituting this into the second condition yields 
\eqn{
-2a -b/2\lambda = \lambda
}
which we can write as 
\eqan{
0&=&\lambda^2+2a\lambda + b/2 \\
0&=&(\lambda+a)^2-a^2+b/2 \\
a^2-b/2&=&(\lambda+a)^2 \\
\pm \sqrt{a^2-b/2} &=& \lambda+a \\
\lambda_{1,2} &=& -a \pm  \sqrt{a^2-b/2} 
}
\subquestion 
You should now have an expression for the eigenvalues that contains a square root. To make sense of it, lets determine for which values of $a$ and $b$ the eigenvalues are complex. (i.e.~when is the term under the root negative such that it the root becomes imaginary.)

\solution
The eigenvalues are complex when
\eqn{
a^2-\frac{b}{2}<0 
}
which is the case when 
\eqn{
2a^2<b
}

\subquestion
Now, assuming the eigenvalues are complex, consider the real part of the two eigenvalues and hence find a condition for stability of the steady state. 

\solution
Since we know that the results of the square root are imaginary in this case the real part is just ${\rm Re}(\lambda)=-a$ and hence the steady state is stable if 
\eqn{a>0.}

\subquestion 
Assuming eigenvalues are real, consider the expression for the leading (i.e.~largest) eigenvalue $\lambda_1$ and hence determine the stability in this case. 
\solution 
Because the we are now assuming that the eigenvalues are real the square has real values and the leading eigenvalue is 
\eqn{
\lambda_{1} = -a + \sqrt{a^2-b/2}. 
}
From this equation we can see that there is no hope for stability if $a<0$ as this would make the first term positive and the square root will be certainly positive. Hence $a>0$ is a necessary condition for stability.

This eigenvalue is negative if 
\eqan{
0 & < & -a + \sqrt{a^2-b/2} \\
a & < & \sqrt{a^2-b/2} \\
a^2 & < & a^2-b/2 \\
0 & < & b/2 \\
0 & < & b 
}
Note that we need squaring both sides of an inequality can produce wrong results if the sides have different signs. However in this case we know the square root is positive and we have already determined that we need $a$ to be positive. Hence if the eigenvalues are real the condition for stability is 
\eqn{a>0,\,b<0}

\subquestion 
Summarize your results. Draw a diagram where $a$ and $b$ are the axis. In this diagram draw lines that correspond to the stability conditions and the points where the eigenvalues change from real to complex. Then color in the area where the steady state is stable. 

\solution
We know 
\begin{itemize}
\item The eigenvalues are complex if $b>2a^2$.
\item If the eigenvalues are complex then the system is stable if $a>0$.
\item If the eigenvalues are real then the system is stable if $a>0$ and $b<0$.
\end{itemize}
We can graphically depict these cases graphically follows:

