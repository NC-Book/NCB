\exercise{The mechanical hydra}{5}
This exercise proposes a hypothetical mechanical device. The first subquestions should be easy to solve. The final one is unsolved but probably solvable. We consider the ODE system
\eqan{
\dot{x}&=&2x-4y \\
\dot{y}&=&3x-5y.
}

\subquestion 
Determine the stability of the steady state at $x=y=0$.
\solution
We compute the Jacobian matrix
\eqn{
{\bf J}=\avecc{2 & -4\\ 3 & -5}. 
}
We compute the characteristic polynomial as 
\eqn{
\left| \avecc{2 - \lambda & -4 \\ 3 & -5 - \lambda} \right| = \lambda^2 +3 \lambda +2
}
Then, we solve 
\eqan{
0&=&\lambda^2 +3 \lambda +2 \\
0&=&(\lambda+3/2)^2 +2 - 9/4 \\ 
1/4 &=& (\lambda+3/2)^2 \\
\pm 1/2 &=& \lambda+3/2 \\
\lambda &=& -3/2 \pm 1/2
}
So, we have two eigenvalues 
\eqn{
\lambda_1 = -1 \qqq \lambda_2=-2. 
}
Since both eigenvalues are negative the steady state is stable. 

\subquestion
Now imagine that we changed this system by clamping some part in place. As a result $y$ becomes fixed at $y=0$ and cannot change anymore, which removes it as a variable. Write the differential equation that describes the dynamics of the remaining system after $y$ has been fixed. 

\solution
As the value of $y$ is not prescribed by the algebraic equation $y=0$ we don't need a differential equation for $y$ anymore. In the differential equation for $x$ we can substitute $y=0$ to remove the remaining $y$, which leaves us with 
\eqn{
\dot{x}=2x
}
\subquestion 
Determine the stability of the steady state in the resulting system. Is the result surprising?

\solution
The system still has a steady state at $x=0$, which isn't very surprising. However if we compute the Jacobian of the new system we find
\eqn{
{\bf J}=2,
}
hence
\eq{
\lambda=2 
}
which shows that the system is now unstable. This result is surprising: We started with a system were $x$ and $y$ sat stably at 0, but if we try to hold $y$ there, then the stability is lost. 

\subquestion
This system exhibits the so-called \emph{hydra effect}. If left alone it settles into a stable steady state, but if we try to hold a certain part in place in this state, then the rest of the system becomes unstable. Your task is to design a mechanical system that exhibits this effect. 

\solution
This is an unsolved question. But there doesn't seem to be a fundamental reason that would make such a system impossible. It would be really interesting to have a mechanical toy that is stable if it isn't held in place.
