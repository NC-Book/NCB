
\exercise{Stability condition for maps}{4}
In the chapter we derived a general condition for the stability of steady states in ODE systems. Now use the same procedure to find a condition for the stability of fixed points in maps. 
\subquestion Start by finding a condition for fixed points in maps. Don't try to guess the result, instead proceed analogously to our approach from the chapter: Define the general one-dimensional map $x_{i+1}=f(x_i)$, consider a fixed point $x^*$, define a deviation $\delta_i$ from the fixed point, find an iteration rule for $\delta_i$, Taylor expand around $x^*$, eliminate all terms that can be eliminated and solve the resulting equation to find the stability condition. (If you get stuck, check the solution)
\solution  
We consider the general one-dimensional map 
\eqn{x_{i+1}=f(x_i)}
Furthermore we suppose that the map has a fixed point $x^*$ such that 
\eqn{
f(x^*)=x^*.
}
We define $\delta_i$ as the deviation from the fixed point
\eqn{
\delta_i=x_i-x^* 
}
and hence 
\eqn{
x_i=x^*+\delta_i
}
We now substitute this equation into the map, which yields 
\eqn{
x^*+\delta_{i+1} = f(\delta_i+x^*)
}
which gives us an iteration rule for the deviation 
\eqn{
\delta_{i+1} =-x^*+f(\delta_i+x^*).
}
We now Taylor expand $f$ around $x^*$ which gives us 
\eqn{
\delta_{i+1} = -x^* + \underbrace{f(x^*)}_{=x^*} +f'(x^*)\delta_i + \underbrace{O({\delta_i}^2)}_{\approx 0}
}
where $f(x^*)=x^*$ according to the definition of fixed points in maps and $O({\delta_i}^2)$ are the terms of quadratic and higher order. For sufficiently small $\delta_i$ the quadratic and higher order terms can be neglected in comparison to the linear term.  

Eliminating the unnecessary terms leads to 
\eqn{
\delta_{i+1} = f'(x^*)\delta_i.
}
This linear map is solved by  
\eq{
\delta_i = (f'(x^*))^i \delta_0
}
We can now see that the deviation $\delta$ gets smaller and smaller $|f'(x^*)|<1$, whereas for $|f'(x^*)|>1$ it becomes larger and larger in maginitude. So $|f'(x^*)|<1$ is the stability condition that we are looking for.  

\subquestion
Now, find the stability condition for multi-dimensional maps $\vec{x_{i+1}}=f(\vec{x_i})$, where $\vec{x_i}$ is a vector of dimension $N$. (Proceed as in (a), when you arrive at a multi-dimensional linear map, solve it using eigendecomposition.)

\solution
We proceed in the same way as for the one.dimensional system. We start by defining $\vec{x^*}$ as the fixed point such that 
\eq{
f(\vec{x^*})=x^*
}
and then introduce the deviation 
\eq{
\vec{\delta} = \vec{x} - \vec{x^*}.
}
By stubstituting $\vec{x} = \vec{x^*} + \vec{\delta} $ into the iteration rule we find 
\eq{
\vec{\delta_{i+1}} + \vec{x^*} = f(\vec{x^*} + \vec{\delta_i}).
}
Now we can Taylor expand the right hand side, which yields
\eq{
\vec{\delta_{i+1}} + \vec{x^*} = f(\vec{x^*}) + {\bf J}\vec{\delta_i} + O(\delta^2)
}
where $\bf J$ is the Jacobian matrix of $f$ evaluated at $\vec{x^*}$ exactly as in the case for differential equations. 

We now recognize that the $f(\vec{x^*})=\vec{x^*}$ on the right cancels with the $\vec{x*}$ on the left. Moreover unless the Jacobian term is zero we can neglect the quadratic terms leaving us with  
\eq{
\vec{\delta_{i+1}} =  {\bf J}\vec{\delta_i}.
}
We have now arrived at a linear map, which we can solve by an eigenvector expansion. So we expand our deviation in the form 
\eq{
\vec{\delta_i} = \sum c_n(i) \vec{v_n}
}
where the $\vec{v_n}$ are eigenvectors of the Jacobian such that 
\eq{
{\bf J} \vec{v_n} = \lambda_n \vec{v_n}
}
Substituting the expansion into the linear map takes us to 
\eq{
\sum c_n(i+1) \vec{v_n} =  {\bf J}\sum c_n(i) \vec{v_n} =  \sum c_n(i) \lambda_n \vec{v_n}.
}
Here it is again time for the best trck ever: As the eigenvectors form a basis the only possible solution is that the first term on the left matches the first term on the right and so on. Hence,
\eq{
c_n(i+1) = \lambda_n c_n(i)  
}
which we can solve for 
\eq{
c_n(i) =  {\lambda_n}^i c_n(0). 
}
Substituting this back would give us the general solution for small perturbations in the neighborhood of fixed points for steady states of maps. 

However more importantly we can now read off the stability condition that we are here for. For the fixed point to be stable the all expansion coefficients must vanish in time, hence all eigenvalues of the Jacobian $\lambda$ must be less than 1. Since the eigenvalues can be complex, the precise condition is that they must be less than 1 in magnitude which we can write as 
\eq{
|\lambda_n|^2 <1 \quad \mbox {for all $n$}.  
}

\subquestion
In Ex.~13.8 we found that the logistic map $x_{i+1}=px_i(1-x_i)$ has fixed points at $x_1^*=0$ and $x_2^*=(p-1)/p$. Determine the stability of these fixed points. 

\solution 
We define the right-hand side of the map as 
\eqn{
px(1-x)=f(x)
}
and compute the derivative 
\eqn{
f'(x) = p-2px 
}
If we now compute the derivative in the first fixed point we find
\eqn{
f'(x_1^*)=f'(0)=p 
}
therefore this fixed point is stable if $|p|<1$ and unstable otherwise. 

For the second fixed point we find
\eqn{
f'(x_2^*) = p-2p \frac{p-1}{p} = p-2(p-1) = 2-p.
}
Hence the second fixed point is stable if $|2-p|<1$ which is the case if 
$1<p<3$.

So in summary: 
\begin{itemize}
  \item For $p\leq -1$ both fixed points are unstable.
  \item For $-1<p<1$ the first (trivial) fixed point is stable while the second (non-trivial) fixed point is unstable 
  \item For $1<p<3$ the non-trival fixed point is now stable while the trivial one is unstable.
  \item  For $p>3$ both fixed points are unstable. 
\end{itemize}

\subquestion 
The equation $5x=x^2$ is easy to solve, but suppose you wanted to solve it by the fixed-point iteration $x_{i+1}={x_i}^2/5$. If you give it a try, you'll find one of the solutions but not the other. Explain why this happens. (Bonus: Come up with a way to nevertheless find the second fixed point by fixed-point iteration.)

\solution 
To explore this we first compute the fixed points, which yields $x_1=0$ and $x_2=-5$.

Now we can study the stability of these fixed points. We consider the derivative
\eq{
\lambda = \left. \frac{\partial x_{i+1}}{\partial x_i} \right|_* = 2 x_i^* /5  
}
For $x^*=0$ we get $\lambda=0$ and since our stability condition for maps is $|\lambda|<1$ the fixed point at $x^*=0$ is stable and our iteration should converge there. 

For the second fixed point $x^*-5$ we find $\lambda=-2$ and hence this fixed point is unstable, which explains why the iteration cannot converge to this point. 

So what can we do? Note that the way we constructed the iteration rule from the original problem is not without alternatives. Instead of turning $5x=x^2$ into $x_{i+1}={x_i}^2/5$ we can also turn it into 
$x_{i+1}=\sqrt{5x_i}$, but no, now $-5$ isn't even a fixed point anymore. Taking square roots of things can make solutions vanish, so let's try one more
\eq{
x_{i+1} = -\sqrt{-5x_i}
}
Now $x^*=-5$ is a fixed point. And when we evaluate the derivative we find 
\eq{
\lambda= -\frac{5}{2\sqrt{-5x^*}} = -\frac{5}{2\sqrt{25}} = - \frac{1}{2}   
}
so the fixed point is stable now and the iteration can converge there. 

The take home message from this is that if your fixed-point iteration does not converge where it should, try different ways of phrasing the iteration rule; that will fix it. 
