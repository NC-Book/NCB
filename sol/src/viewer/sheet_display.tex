\documentclass{sheet}

\showsheet{3}   % Change this to display a different sheet


\exercise{Averages revision \label{exSWaverage}}{1}
Compute the averages of the following sequences of numbers: a) $3,7,5$;  b) $1,3,6,1,2,5$;  c) a sequence containing 13 times the number 3 and 7 times the number 17. 

\solution
We find the solution by adding up all the numbers and dividing by the number of numbers that we added up. This means for part (a)
\eq{
\frac{3+7+5}{3} = \frac{15}{3}=5
}
For part (b)
\eq{
\frac{1+3+6+1+2+5}{6}= \frac{18}{6}=3
}
For part (c) we can save us some work by using multiplication to add up the numbers rather than adding them up term by term 
\eq{
\frac{13\cdot 3 + 7 \cdot 17}{13+7}= \frac{39+119}{20} = \frac{158}{20}=7+\frac{18}{20}=7.9
}

%%%%%%%%%%%%%%%%%%%%%%%%%%%%%%%%%%%%%%%%%%%%%%%%%%%%%%%%%%%%%%%%%%%%%%

\exercise{Logarithm revision\label{exSWlog}}{1}
Note that $\log_{10}(2)\approx 0.3$. Use this to solve the following equations without a calculator: a) $20=10^x$, b) $4000=10^x$ c) $2=1000^x$ d) $5=10^x$ e) $2^x=10^9$. If you can't do it, check out the explanations in the solutions immediately.

\solution
The logarithm $x=\log_a{y}$ is (by definition) the solution to \eq{y=a^x}. Some useful consequences of this definition are 
\eqa{
\mbox{i)} &\quad& a^{\log_a(x) } = x = \log_a(a^x) \\
\mbox{ii)} &\quad& \log_a(xy) = \log_a(x) + \log_a(y) \\
\mbox{iii)} &\quad& \log_a(x^y) = y\log_a(x)  \\
\mbox{iv)} &\quad& \log_a(x) = \log_b(x) / \log_b(a) \\ 
\mbox{v)} &\quad& \log_a(1/x) = -\log_a(x) \\
\mbox{vi)} &\quad& \log_x(x) = 1
}

To solve (a) we use
\eq{
x=\log_{10}(20) = \log_{10}(2)+\log_{10}(10) \approx 0.3 + 1 = 1.3
}
Note that the second term, where we used rule ii, and then rule vi.

For part (b) we can do the following:
\eq{
x=\log_{10}(8000)=\log_10(8)+\log_{10}(1000) = \log_{10}(2^3)+\log_{10}(10^3)\approx 3\cdot 0.3+3 \approx 3.9 
}
where we used rule ii, then wrote the numbers as powers to apply rule iii.

Now for part (c), we need rule iv
\eq{
x=\log_{1000}(2)=\frac{\log_{10}(2)}{\log_{10}(1000)}\approx \frac{0.3}{3}=0.1 
}

We do part (d) almost in the same way
\eq{
x=\log_{10}(5)= \log_{10}(10/2) = \log_{10}(10)-\log_{10}(2)\approx 1-0.3 = 0.7
}
The key idea here is to write 5 as 10/2 and then use rules ii and v.

Finally for (e) we start with a change of basis (rule iv)
\eq{
x=\log_{2}(10^9)= \frac{\log_{10}(10^9)}{\log_{10}(2)}\approx \frac{9}{0.3} = \frac{90}{3} = 30
}

%%%%%%%%%%%%%%%%%%%%%%%%%%%%%%%%%%%%%%%%%%%%%%%%%%%%%%%%%%%%%%%%%%

\exercise{Diameter of a specific network\label{exSWsimplediameter}}{2}
Consider a network consisting of a central hub of degree 9 that connects to 9 nodes of degree 1. Compute the diameter according to both definitions (do not use the approximation, it won't work well for such a small and heterogeneous network)  

\solution
If we use the classical definition `longest shortest path', we get the answer 2, immediately. If we use the average path length definition we note that the distance from the central node to one of the 9 peripheral nodes is 1. The distance from one of the 9 peripheral nodes to one the central node is also 1. and the distance between two peripheral nodes is 2. Sine there are $9\cdot8=72$ distances between peripheral nodes we get the mean path length
\eq{
D = \frac{9\cdot1+9\cdot 1 + 72 \cdot 2}{9+9+72} = \frac{80+72}{80}=1.9 
}

%%%%%%%%%%%%%%%%%%%%%%%%%%%%%%%%%%%%%%%%%%%%%%%%%%%%%%%%%%%%%%%%%%%%%

\exercise{Large abstract network \label{exSWabstract}}{2}
A network has  $N=20,000$ nodes mean degree of $z=10$ and and no significant clustering coefficient ($c\approx 0$.) Estimate the diameter of this network. 

\solution
We can use our formula
\eqa{
D&=& \log_{z}(N) \\
 &=& \log_{10}(20,000) \\
 &=& \log_{10}(2)+\log_{10}(10^4) \\
 &\approx& 4.3
}

%%%%%%%%%%%%%%%%%%%%%%%%%%%%%%%%%%%%%%%%%%%%%%%%%%%%%%%%%%%%%%%%%%%%%

\exercise{Social diameter of Iceland\label{exSWiceland}}{2}
Iceland has a population of 340,000. Let's assume that the mean degree of the acquaintance network in Iceland is $z=120$ and the clustering coefficient is $c=0.2$. Estimate the social diameter of Iceland. 

\solution
We can use our formula for diameter with clustering
\eqa{
D&=&\log_{(1-c)z}((1-c)N) \\
 &=& \log_{96}(272,000) \\ 
 &\approx& 2.74
}

%%%%%%%%%%%%%%%%%%%%%%%%%%%%%%%%%%%%%%%%%%%%%%%%%%%%%%%%%%%%%%%%%%%%

\exercise{Local clustering coefficient\label{exSWlocal}}{3}
In a friendship network we find $n_{--}=45000$ three node chains and $n_{\Delta}=1500$ three-cycles (triangles). 
\subquestion Compute the clustering coefficient. 

\solution
We compute the clustering coefficient using the formula from the chapter
\eq{
c=\frac{3n_\Delta}{n_{--}} = \frac{4.500}{45.000} =0.1 = 10\%.  
}

\subquestion Ali is a node in the network, he has $k=17$ friends. How many friendship links $f$ do you expect to find between Ali's 17 friends? 

\solution
Ali has $k= 17$  friends that means that the number of distinct pair of friends between friendships could exist is 
\eq{
n = \frac{17\cdot{}16}{2} = 136.
}
This is also the number of three-node-chains that have Ali in the center. 

From the clustering coefficient we know that 10\% of these chains are closed triangles, so our best guess is that ca.~14 friendships exist among Ali's friends.

\subquestion Use the insights gained from this example to write a general formula that can be used to estimate the clustering coefficient $c$ from $k$ and $f$.    

\solution
The final part asks for a general formula. We start from the observation that the clustering coefficient is the probability that two of friends of a node (ego) are also mutually friends. We can write this as 
\eq{
\label{testtest}
c=\frac{f}{n}, 
}
where $n$ is the number of three-node-chains centered with ego in the center or in other words the number of ways in which we can pick 2 two people from ego's set of friends. We can compute this number as
\eq{
n=\frac{k(k-1)}{2},
}
where $k$ is the number of ego's friends. Substituting this relationship into the equation above yields the desired formula
\eq{
c=\frac{2f}{k(k-1)}. 
}
You can use this formula for a quick rough estimate the clustering coefficient of the human friendship network, based on your number of friends and the number of friendship links between them. 


%%%%%%%%%%%%%%%%%%%%%%%%%%%%%%%%%%%%%%%%%%%%%%%%%%%%%%%%%%%%%%%%%%%%%

\exercise{A small example \label{exSWbig}}{3}
\hint{Hint}{Trying to count the three-node chains in this net is a pain. Perhaps some maths can help.}Find the clustering coefficient $c$ and mean degree $z$ of the following network:
\pic{bignet}{0.65}

\solution
In this network we can just count the triangles to find
\eq{
n_{\Delta}=7
}
I am pretty much sure that I just don't have the focus that would be necessary to count the three-node chains and arrive at the right result. Fortunately \exref{exSWlocal} gives us an idea, how to do the counting more efficiently. The number of three node chains centered on a node of degree $k$ is $k(k-1)/2$. So we can make the following table: 
\begin{center}
\begin{tabular}{l | c  c  c  c  c  c}
Degree $k$ & 1 & 2 & 3 & 4 & 5 & 6 \\\hline
Chains per node & 0 & 1 & 3 & 6 & 10 & 15 \\
Number of nodes & 3 & 0 & 3 & 4 & 0 & 1  \\
Product         & 0 & 0 & 9 & 24 & 0 & 15 
\end{tabular}    
\end{center}
In the top line I have computed the number of three node chains that are centered on a node of given degree. In the second line I just counted how many nodes of the respective degree exist in the network. Multiplying these two lines gives us the third line which contains the number of three node chains centered on nodes of a given degree. 
For example there are three nodes of degree 3 (F,H,J) and every node of degree 3 is at the center of 3 three-node chains, so this gives us 9 three-node chains that are centered on nodes of degree 3. SO in total there are 
\eq{
n_{--}= 9+15+24 = 48 
}
three-node chains in the network. We can now compute the clustering coefficient 
\eq{
c= \frac{3\cdot 7}{48} = \frac{7}{16}=0.4375
}
Even though this network seemingly contains a lot of triangles. it has a clustering coefficient of less than 0.5. 

%%%%%%%%%%%%%%%%%%%%%%%%%%%%%%%%%%%%%%%%%%%%%%%%%%%%%%%%%%%%%%%%%%%%

\exercise{Three hop rule\label{exSWthreehops}}{3}
In the USA the Department of Homeland Security routinely collects and stores communication metadata as part of its anti-terrorism efforts. The use of this data is governed by the so-called three-hop-rule, which means that if there is a terrorism suspect investigators can access the data up to 3 links away from the suspect. We do not know how extensive the database is but it seems fair to assume that it is not very different from the facebook network. So let's assume $z=200$, $c=0.15$ and $N=10^{10}$. Compute the number of records that can be accessed in one investigation. Then, suppose there are 10.000 investigations per year. How often will your records be accessed in a year?

\solution
We know that the number of nodes 3 steps from the focal node is 
\eq{
n_3 = z^3(1-c)^2 = 5,780,000
}
We could add the number of nodes at distance 1 and 2, but these 
(34,000 and 200, respectively) hardly matter. Let's say one investigation looks at 5.8 million, that means there is a 0.58\% chance that my records are accessed in an investigation. So the expectation value for the number of times my records are accessed is 
\eq{
0.0058\cdot 10,000 = 58
}
so approximately 58 times a year. 

%%%%%%%%%%%%%%%%%%%%%%%%%%%%%%%%%%%%%%%%%%%%%%%%%%%%%%%%%%%%%%%%%%%%

\exercise{Shapers\label{exSWshapers}}{3}
I play an augmented reality game, that has ca.~$5\cdot{}10^5$ players worldwide. Through the game I have made many friends and I regularly meet with 20 of them in the real world. Among these friends is a group of ca.~7 people, which know each other well and meet weekly. Among the rest only two know meet each other in real life. Part of the game is to pass virtual items around that behave like physical objects, i.e.~to give an item to a player the two of you must meet in the real world. Often items need to reach a specific target person. It is therefore interesting to ask how many times an item has to be handed over to reach its destination. In other words, what is average path length in the network where nodes are players and links represent physical meetings.     

\solution
The first step to the solution is to estimate the clustering coefficient. Without any additional information we can only work with my personal experiences. First let's ask how many friendship links exist between my friends. We know that I have one pair of friends who meet each other, so that's one. There is also the group of 7 people, who all meet each other mutually. This gives us another $7\cdot{}6/2=21$ links between my friends for a total of $f=22$. 
Using the reasoning from Ex.~5.6 we can estimate the clustering coefficient as 
\eq{
c=\frac{2f}{k(k-1)}
}
where $k=20$ is the number of the friends I meet. This yields a clustering coefficient of $c\approx 0.12$.
 We can now compute $1-c\approx 0.88$ and use this to estimate the average path length
\eq{
D=\log_{0.88\cdot20}(0.88\cdot 5\cdot{}10^5) \approx 4.5
}
so we should typically be able to get an item to the target person who needs to have it within a few hops. All of this assumes of course that I am an average player (which isn't far from the truth). 

%%%%%%%%%%%%%%%%%%%%%%%%%%%%%%%%%%%%%%%%%%%%%%%%%%%%%%%%%%%%%%%%%%%%%

\exercise{A real problem\label{exSWlabor}}{3}
In most western countries child labor has been illegal since the worker's rights movement in the beginning of the 20th century. Some countries also make it illegal for companies to do business with suppliers that use child labor. Suppose a new law is passed that also makes it illegal to have suppliers who have suppliers that use child labor. You know that your company does not use child labor, nor does any of your suppliers. But you are not sure about their suppliers. Assuming that there are $10^6$ companies and 1000 of these use child labor. A typical company has about 600 suppliers. Given these numbers, do we have to worry that a supplier of your suppliers uses child labor?

\solution
As we don't know the clustering coefficient (and it is probably relatively low) our estimate is that the number of our supplier's suppliers could be a s high as 36.000. As one company in 1.000 uses child labor, we would expect that 36 of our supplier's suppliers use child labor. 

This is an important real world problem. For a firm, mapping all of their supplier's suppliers is very hard as it requires the supplier's cooperation, however not knowing who your supplier's suppliers are can carry significant legal, reputational and business disruption risks. The network of business relationships turns out to be a very small world, which means that disturbances anywhere in the world can disrupt production here. Their is an increasing demand for methods from complex systems theory to manage and regulate this highly complex network. 

The bigger picture here is that the global supply network has such a small diameter, that a disturbance anywhere is felt everywhere. 

%%%%%%%%%%%%%%%%%%%%%%%%%%%%%%%%%%%%%%%%%%%%%%%%%%%%%%%%%%%%%%%%%%%%%

\exercise{Computer network\label{exSWcomputer}}{4}
A computer network consists of 5 servers and 100 workstations. Each server is connected to all other servers and to 20 of the workstations. The workstations have no further connections. Find the exact value of the average path length, and compare it to approximations, with and without correction for clustering coefficient. (You will find that the approximation performs exceptionally poorly in this case. A brief explanation why this happens is given in the solution.)  

\solution
We can compute the following distances (S=Server, W=Worstation):
\begin{center}
\begin{tabular}{l c c}
Communication & distance & number of paths \\\hline
S to S & 1 & $5\cdot 4/2 = 10$ \\
S to W on that server & 1 & $5\cdot 20 = 100$ \\
S to W on other server & 2 & $ 5 \cdot 80 = 400$ \\
W to W on same server & 2 & $5 \cdot 20 \cdot 19 /2 = 950$ \\
W to W on different server & 3 & $5\cdot 20 \cdot 80/2=4000$
\end{tabular}    
\end{center}
So the diameter is 
\eq{
D_{]rm Exact}=\frac{1\cdot (10+100))+2\cdot (400+950) + 3 \cdot 4000}{10+100+400+950+4000}=\frac{14810}{5460}\approx 2.712
}
A good way to make sure that we have not forgotten any type of path, is to compute the the total number of shortest paths in a different way. There are in total 105 computers in the network, so the total number of source-destination pairs is 
\eq{
105\cdot 104/2 =  5460
}
which is also the denominator that we found by adding up all the individual types of shortest paths considered in the table. This shows that we dd not forget to consider paths. 

To use our diameter estimate we first need the mean degree. There are 10 links between the 5 servers and 100 links between servers and workstations, so our network has 110 links in total between its 105 nodes. So the mean degree is 
\eq{
z= \frac{220}{105} \approx 2.095
}
Using our simple diameter estimate we would expect that the average path length is 
\eq{
D_{\rm Est}=\log_{2.095}(105) \approx 6.44 
}
which more than twice the actual value. SO let's try the formula with the clustering coefficient correction. To do so let's first count the triangles in the network. The number of three-node-chains between servers is $5\cdot 4 \cdot 3 / 2 = 30$ as all of these are part of triangles we can simply find the number of triangles by dividing by 3 to obtain \eq{n_{\Delta}=10.}
To find the total number of three-node chains we have to add the all the three node chains between servers and workstations connected to other servers ($5\cdot 4 \cdot 20=400$) and chains between workstations on the same server ($5 \cdot 19 \cdot 20/2=950$), hence 
\eq{
n_{--} = 30 + 400 + 950 = 1380. 
}
Thus the clustering coefficient is almost zero, 
\eq{
c=\frac{3_{n_{\Delta}}}{n_{--}} = \frac{30}{1380}\approx 0.02
}
At this low value we don't expect the clustering coefficient to expect the diameter estimate very much. Indeed we find 
\eq{
D_{\rm Est2}=\log_{(1-c)z ((1-c)N)} \approx log_{2.05}(103) ]approx 6.46
}
So the correction due to the clustering coefficient is very minor and takes our estimate even a bit further from the truth. 

Why does the estimate perform so poorly in this case? Consider that in our derivation of the estimate we assumed that the networks are reasonably randomly connected. However, our computer network is a very non-random network. In fact, the network that we considered is not a typical example of a network with 105 nodes and 110 links. Instead among the gazillion networks that can be constructed out of 105 nodes and 110 links it is among a very small number of topologies that have a very low diameter. 

Of course this is no accident, we build computer networks in this way precisely because having central highly connected servers brings down the diameter substantially and thus allows data to be exchanged efficiently. 

We will understand the mathematical reason why this happens in the next chapter. 

%%%%%%%%%%%%%%%%%%%%%%%%%%%%%%%%%%%%%%%%%%%%%%%%%%%%%%%%%%%%%%%%%%%%%

\exercise{A mean degree puzzle\label{exSWmean}}{4}
A connected network has a mean degree of 
$z=1.99901234568\ldots$. How many nodes does the network contain?

This is a puzzle! Start by drawing some small networks and calculate $z$. How do you have to change the network to bring $z$ closer to the desired value, while remaining connected?

\solution
If we follow the advice we find very quickly that every small network that we draw will have a mean degree that is too high, unless it is a tree. In fact you might notice that every connected network that contains a single cycle has a mean degree of exactly 2 and every network that contains more than one cycle has a mean degree greater than two. So the network we are looking for must be a tree. 

We also noticed that in a tree of $N$ nodes the number of links is $K=N-1$. 
(this can be easily proved by induction, the tree with 1 node contains no links, to add another node to the tree we need exactly one link, etc.)
Hence the mean degree of a tree is 
\eq{
z=\frac{2K}{N} = \frac{2(N-1)}{N} 
}
In the exercise we are asked to find the number of nodes for a given mean degree so let's solve for $N$
\eqa{
z&=&2(N-1)/N \\
Nz&=&2(N-1) \\
Nz&=&2N-2 \\
N(z-2)&=&-2 \\
N&=&2/(2-z)
}
Substituting the mean degree from the question we get
\eq{
N=\frac{2}{2-1.99901234568}\approx 2025.0000
}
so the answer is 2025, the year in which this book was published.  

\solutionend













