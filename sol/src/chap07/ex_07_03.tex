

\exercise{Degree distribution of ER graphs \label{exRGdegreedist}}{2}
We discovered that the degree distribution of ER graphs can be written as $p_k=z^{k}\exp{-z}/{k!}$.

\subquestion 
Show that $p_k$ is a correctly normalized probability distribution. (i.e.~$\sum p_k=1$)

\solution
We compute 
\eqa{
\sum p_k &=& \sum \frac{z^{k}\exp{-z}}{k!} \\
  &=&  \exp{-z} \sum \frac{z^{k}}{k!} \\
  &=&  \exp{-z}\exp{z} = 1 
}
as it should be. In the second step we were able to pull $\exp{-z}$ out of the sum as it does not depend on $k$. Thereafter we used the 
$\sum x^k /k! = \exp{x}$ from the chapter. 

\subquestion
Compute the mean degree from $p_k$ to confirm that it is $z$.

\solution
In this case we compute
\eq{
\sum kp_k = \sum k\frac{z^{k}\exp{-z}}{k!} = \exp{-z} \sum k\frac{z^{k}}{k!} 
}
Again we pulled $\exp{-z}$ out of the sum, however we can't do the same to $k$ because it depends on $k$ (in fact, it is $k$, and outside the sum we wouldn't know what it's value is). 
Note however that the last factor of the factorial in the denominator is also $k$, so unless $k=0$ we can cancel these $k$'s. Now our sum starts with a $k=0$ term but this term is actually 0. So we can just ignore it, start the sum at $k=1$ and do the cancellation of the $k$
\eq{
\sum kp_k = \exp{-z} \sum_{k=1}^\infty \frac{z^{k}}{(k-1)!} 
}
If we haven't done it already, now is the time to think how we are going to turn this equation into something nice. We need some way to 
take care of the sum. Our best shot is again to apply $\sum x^k/k!=\exp{x}$. So we would like to have a $k!$ in the denominator, not $(k-1)!$. We could try expanding the whole fraction by $k$ but that would just undo the simplifiation we just did. Instead, let's try shifting the index. 
\eq{
\sum kp_k = \exp{-z} \sum \frac{z^{k+1}}{k!} 
}
where the summation starts again at zero. We are now almost where we wanted to go but there is a factor of $z$ too much in the numerator. Fortunately we can just pull this out of the sum
\eq{
\sum kp_k = z \exp{-z}  \sum \frac{z^k}{k!} 
}
Now we can apply the series expansion backwards to arrive at the result
\eq{
\sum kp_k = z \exp{-z} \exp{z}= z
}
The mean degree is $z$ as expected. 

\subquestion
Finally, compute the excess degree distribution $q_k$ and the mean excess degree $q$.

\solution
This one is fun: We just substitute in the Poisson distribution into our formula for $q_k$ from the previous chapter and simplify
\eqa{
q_k &=& \frac{(k+1)}{z}p_{k+1} \\
  &=& \frac{(k+1)}{z} \frac{z^{k+1}\exp{-z}}{(k+1)!} \\
  &=& (k+1) \frac{z^{k}\exp{-z}}{(k+1)!} \\
  &=& \frac{z^{k}\exp{-z}}{k!} = p_k 
}
The excess degree distribution is exactly identical to the degree distribution. Therefore, the mean excess degree must also be identical to the mean degree,
\eq{
q=\sum k q_k = \sum k p_k = z 
}

