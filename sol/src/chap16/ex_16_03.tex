


\exercise{Abstract sensitivity example}{2}
The following model is quite complicated:
\eqn{
\dot{x} = -x^4 + 2px^3 + 2p^2x - px^2 + x - 2p
}
It would be a pain to compute the steady states by hand. However, 
suppose that based on your understanding of the system you suspect that 
for $p^*=1$ there is a steady state at $x^*=2$. 
\subquestion
Verify that $x^*=2$ is indeed a steady state at $p^*=1$.
\solution
Substituting $p=1$ yields
\eq{
\dot{x} = - x^4 + 2x^3 + 2x  - x^2 + x - 2
}
and then substituting $x=2$ we find
\eqn{
\dot{x} = -2^4 + 2\cdot 2^3 + 2\cdot 2 - 2^2 + 2 - 2 = 0
}
so this is indeed a steady state. 

\subquestion
Use the equation for the sensitivity of steady states from the \vs{chapter}{lecture} to compute how the steady state is affected if we change $p$ a little bit. 

\solution
We need 
\eqa{
f_{\rm x}(x^*,p^*)&=& \left. \frac{\partial}{\partial x} (-x^4 + 2px^3 + 2p^2x  - px^2 + x-2p)\right|_* \\
   &=& \left. (-4x^3 + 6px^2 + 2p^2  - 2px + 1)\right|_* \\
   &=&  -4\cdot 2^3 + 6\cdot 1 \cdot 2^2 + 2\cdot 1^2  - 2\cdot 1 \cdot 2 + 1 \\
   &=&  -32 + 24 + 2  - 4 + 1  \\
   &=&  -9
}
and also 
\eqa{
f_{\rm p}(x^*,p^*)&=& \left. \frac{\partial}{\partial p} (-x^4 + 2px^3 + 2p^2x  - px^2 + x - 2p)\right|_* \\
&=& \left. ( 2x^3 + 4px  - x^2 - 2)\right|_* \\
&=&  2\cdot 2^3 + 4\cdot 1\cdot 2  - 2^2 - 2 \\
&=&  16 + 8  - 4 - 2 \\
&=& 18
}
We can now compute 
\eq{
\left.\frac{{\rm d} x^*}{{\rm d} p} \right|_* = -\frac{f_{\rm p}}{f_{\rm x}} = -\frac{18}{-9} = 2  
}
In words: If we start from the steady state at $p^*=1$, $x^*=2$ and then change $p$ by a small amount $\rho$ we expect the steady state to shift by $2\rho$.

[Bonus: Notice anything funny about the calculations in this exercise. There is a little bit more to discover here.]
