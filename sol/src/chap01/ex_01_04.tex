

%%% Exercise 4 %%%


\exercise{Return to Moravia}{2}
 Solve the Moravian example with a slightly modified distance matrix that is now given by
 \eqn{
{\bf D} = \left(\begin{array}{c c c c c} 
 0  & 137 & 100 &  74 &  77 \\
137 &  0  &  75 &  76 & 198 \\
100 &  75 &  0  &  51 & 121 \\ 
 74 &  76 &  51 &  0  & 151 \\
 77 & 198 & 121 & 151 &  0  
\end{array}\right)
 }
 and the nodes are 
 \eqn{
1: {\rm B}, \, 2: {\rm O}, \, 3: {\rm L}, \, 4: {\rm Z}, \, 5: { \rm J}.
}

% solution %

\solution
The solution network has the edge set 
\eq{
E=\{(\rm L,Z),(B,Z),(O,L),(B,J)\}
}
we find it by
\begin{enumerate}[label=\arabic*.]
    \item Try (L,Z) [51km] -- accept
    \item Try (B,Z) [74km] -- accept
    \item Try (L,O) [75km] -- accept
    \item Try (O,Z) [76km] -- reject
    \item Try (B,J) [77km] -- accept 
\end{enumerate}
Of course just drawing the solution network is also an acceptable answer to this question, but also writing down these steps makes it easier to discuss the solution and identify the nature of the mistake if there is one.  

