
%%% Exercise 8 %%%

\exercise{Another network in Moravia}{3}
Let's revisit the Moravian example, described by the distance matrix 
\eqn{
{\bf D} = \left(\begin{array}{c c c c c} 
 0  & 137 &  63 &  74 &  77 \\
137 &  0  &  75 &  76 & 198 \\
 63 &  75 &  0  &  51 & 121 \\ 
 74 &  76 &  51 &  0  & 151 \\
 77 & 198 & 121 & 151 &  0  
\end{array}\right)
}
where the nodes are again 
\eqn{
1: {\rm B}, \, 2: {\rm O}, \, 3: {\rm L}, \, 4: {\rm Z}, \, 5: { \rm J}.
}
This time we assume that when we get to work there is already a power line from B to Z. Find which additional lines need to be built such that all cities are connected and the length of additional lines built is minimal. 

% solution %

\solution
We can apply Kruskal's algorithm, but line (B,Z) is already there (or alternatively, we could say it has cost 0). So what we do is
\begin{enumerate}[label=\arabic*.]
\item Place (B,Z)
\item Try (L,Z) [52km] -- accept
\item Try (L,B) [63km] -- reject
\item Try (L,O) [75km] -- accept
\item Try (O,Z) [76km] -- reject
\item Try (B,J) [77km] -- accept
\end{enumerate}
So, the final edge set is 
\eq{
E= \{{\rm (B,Z),(L,Z),(L,O),(B,J)}\}.
}

