
\exercise{Some quick calculations  \label{exAttackQuick}}{2} 
The following questions can be answered with very short calculations:

\subquestion 
A configuration model network has mean excess degree $q=5$. We attack it by random node removal. What proportion of nodes doe we have to remove to break the giant component?

\solution
The giant component breaks at $q_{\rm a}=cq=1$, hence the proportion of surviving nodes is $c=1/5$ which implies $r=1-c=4/5$. So, we need to remove 80\% of the network.   

\subquestion 
A configuration model network has mean degree $3$ and mean excess degree $7$. We attack it by removing nodes of degree 10.  What proportion of nodes do we need to remove to break the giant component?

\solution 
The attack generating function in this case is 
\eq{
R=r_{10}x^{10}
}
We also know that the reduction in excess degree is 
\eq{
\delta= \frac{R''(1)}{z}= \frac{90}{3} r_{10} = 30 r_{10}
}
Since we need to reduce $q$ to 1 to break the giant component, we need $\delta=6$ to break the giant component. Solving 
\eq{
6=30r_{10}
}
yields 
\eq{
r_{10}=\frac{1}{5},
}
hence we need to remove 20\% of all nodes (assuming that this many nodes of degree 10 exist in the network)

\subquestion
Show that removing nodes of degree 0 or 1 from a configuration model network does not change the mean excess degree. 

\solution
We know from the chapter that the change in the mean excess degree from a targeted attack is 
\eq{
\delta=\frac{R''(1)}{z}
}
A general attack that removes some nodes of degree 0 or 1 but no other nodes is described by 
\eq{
R=r_0+r_1x
}
Since there is no quadratic term, computing the second derivative yields  $\delta=0$. 
