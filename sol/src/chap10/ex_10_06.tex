
\exercise{Abstract viral attack}{3}
Consider a configuration model network with degree distribution $p_k=0.8\delta_{k,2}+0.2 \delta_{k,8}$. A viral attack can spread over 40\% of the links in the network. Is there still a giant component after the nodes affected by the viral attack have been removed? 

\solution 
We follow the procedure established in the chapter and start by computing some basic network properties. We start by writing the degree generating function 
\eq{
G=\sum p_k x^k = 0.8x^2 + 0.2x^8
}
and then compute the mean degree 
\eq{
z=G'(1)=1.6+1.6=3.2
}
We con now compute the excess degree generating function 
\eq{
Q=\frac{G'}{z}=\frac{1.6x+1.6x^7}{3.2} = 0.5x+0.5x^7 
}
and the mean excess degree 
\eq{
q=Q'(1)=4 
}
The attack can spread over 40\% of the links so $1-w=0.4$ and hence $w=0.6$.

Before we dive the main part of the calculation, first let's quickly check if there is a giant conducting component at all. We compute 
\eq{
q_{\rm c} = (1-w)q = 0.4 \cdot 4 = 1.6>1
}
Since the mean excess degree of the conducting network is greater than 1 there is a giant conducting component. 

Now we are ready for step 2 in our viral attack algorithm. To compute the size of the giant conducting component we need to construct pruning function that removes the non-conducting links
\eq{
A_{\rm c}=0.4x+0.6
}
Now we can compute $v_{\rm c}$ for the giant conducting component by solving 
\eq{
v_{\rm{c}} = Q(A_{\rm c}(v_{\rm c}))  
}
For the present network this is
\eq{
v_{\rm c}= 0.5(0.4x+0.6)+0.5(0.4x+0.6)^7
}
Starting from $v_{\rm c}=0.5$ and iterating numerically yields
\begin{center}
    0.5, 0.5048576, 0.5076249095 \ldots 0.5113848791 
\end{center}

In step 3 of the viral attack algorithm we the proportion of all links that are not in the giant conducting component
\eq{
y=A_{\rm c}(v_{\rm c})=0.4 v_{\rm c} + 0.6 \approx 0.8045539517
}
so only about 20\% of the links are in the gcc. Now that we know this we can compute the size of the gcc, i.e.~the proportion of the links that will be removed if an attack on the giant conducting component occurs as
\eq{
r=s_{\rm c}=1-G(y)= 0.8y^2 + 0.2y^8 \approx 0.5529589268
}
This tells us that starting the viral attack in a random node has 55\% chance to cause the removal of 55\% of the network. 

If those 55\% of the nodes were chosen randomly then the mean excess degree after the attack would be $(1-r)q\approx 1.788>1$, but the random attack does not chose its targets randomly. So, instead we compute the mean access after the attack as follows 
\eqa{
q_{\rm a} &=& y^2 Q'(y)\\
  &=& y^2 (0.5+3.5y^6) \\
  &=& 0.5 y^2 + 3.5 y^8 \\
  &\approx& 0.9381358939 < 1
}
which shows that the giant component in the network under consideration would not survive the attack. 
