
\exercise{Targeted attacks in general \label{exAttackTargeted}}{3}
In this exercise we derive some results on degree-targeted attacks in general. Consider a configuration model network, described by a degree generating function $G$, that is subject to an attack described by a removal function $R$. Try to do the calculations below in the forward direction (deriving the result) instead of the backward direction (proving the result is true).

\subquestion Show that the degree distribution of the surviving nodes after the first step of the attack is generated by $G_{\rm h}=\frac{G-R}{c}$.

\solution
We can start by recalling that the number of nodes with degree $k$ is 
\eq{
n_k =  N p_k. 
}
Moreover the number of nodes of degree $k$ that are removed in the attack are $N r_k$. So the number of nodes of degree $k$ that survive the first step of the attack are
\eq{
n_{k,{\rm h}} = N (p_k-r_k)
}
We define 
\eq{c=1-r=1-\sum r_k} 
as the proportion of nodes that survive the attack. Hence the total number of nodes in the network after the attack is 
\eq{N_{\rm h} = cN}
We can now compute the proportion of nodes of degree $k$ after the first step of the attack as 
\eq{
p_{k,{\rm h}} = \frac{n_{k,{\rm h}}}{N_{\rm h}} = \frac{N (p_k-r_k)}{cN} = \frac{ p_k-r_k}{c}
}
Now we multiply both sides of the equation with $x^k$ and sum over all $k$,
\eq{
\sum p_{k,{\rm h}} x^k =  \sum \frac{ p_k-r_k}{c} x^k \\
}
Finally using the definitions of the generating functions $G_{\rm h} = \sum p_{k,{\rm h}} x^k $, $G=\sum p_k x^k$, and $R=\sum r_k xk$, we arrive at the desired result
\eq{
G_{\rm h} = \frac{G-R}{c}
}

\subquestion Show that the mean degree after the attack is $z_{\rm a}=z\tilde{c}^2/c$.

\solution 
First we recall that the degree generating function after the attack is 
\eq{
G_{\rm a} = \frac{G(\tilde{A})-R(\tilde{A})}{c} 
}
where 
\eq{
\tilde{A} = \tilde{r} +\tilde{c}x,
}
and $\tilde{r}$, $\tilde{c}$ are the removed and surviving proportions of endpoints respectively. They are defined by  
\eq{
\tilde{r} = \frac{R'(1)}{z},\qqq \tilde{c}= 1-\tilde{r}
}

Let's compute $z_{\rm a}$ as 
\eq{
z_{\rm a} = G_{\rm a}'(1) = \frac{G'(\tilde{A}(1))-R'(\tilde{A}(1))}{c}\tilde{A}'(1)  = \frac{G'(1)-R'(1)}{c}\tilde{c} =  \frac{z-R'(1)}{c}\tilde{c}
}
At this point the $R'(1)$ is bothering us a little bit, so how can we get rid of it? We have last seen it in the definition of $\tilde{r}$ and using this definition we can replace 
\eq{
R'(1) = z \tilde{r}
}
and hence 
\eq{
z_{\rm a} = \frac{z-z\tilde{r}}{c}\tilde{c} = \frac{z(1-\tilde{r})\tilde{c}}{c} = z \frac{\tilde{c}^2}{c},
}
which is the desired result. 

\subquestion Show that the excess degree distribution after the attack is generated by $Q_{\rm a}=(G'(\tilde{A})-R'(\tilde{A}))/\tilde{c}z$.

\solution
We can now compute the excess degree generating function as 
\eq{
Q_{\rm a} = \frac{G'_{\rm a}}{z_{\rm a}} = \frac{G'(\tilde{A})-R'(\tilde{A})}{cz_{\rm a}}\tilde{A}' = (G'(\tilde{A})-R'(\tilde{A}))\frac{\tilde{c}}{z\tilde{c}^2}  = \frac{G'(\tilde{A})-R'(\tilde{A})}{z\tilde{c}} 
}

\subquestion Show that the mean excess degree after the attack is
$q_{\rm a}=q-R''(1)/z$.

\solution 
Since we already know the generating function $Q_{\rm a}$ we can compute 
\eq{
q_{\rm a} = Q_{\rm a}'(1) = \frac{G''(\tilde{A}(1))-R''(\tilde{A}(1))}{z\tilde{c}} \tilde{A}'(1) = \frac{G''(1)-R''(1)}{z\tilde{c}} \tilde{c} 
= \frac{G''(1)-R''(1)}{z}
}
now we only need to recognize
\eq{
\frac{G''(1)}{z} = q,
}
which leads us to the desired result
\eq{
q_{\rm a} = q - \frac{R''(1)}{z}.
}
