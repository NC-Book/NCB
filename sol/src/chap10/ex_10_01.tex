\exercise{Polynomial Long Division}{1}\label{exAttackPolyLong}In the chapter we needed to solve $v=(0.8v+0.2)^3$. Use the known solution $v=1$ to reduce this equation to a quadratic polynomial by polynomial long division. [This is not essential, but a neat trick. Just check the solution if you haven't seen it before.]

\solution
To get started we bring the condition into a nicer form. write the decimals as fractions, multiply out the cubic function, and bring everything to one side. 
\eqa{
0&=& \left(\frac{4}{5}v+\frac{1}{5}\right)^3 -v \\
 &=& (4v+1)^3 -125v \\ 
 &=& 64v^3+48v^2+12v+1-125v \\
 &=& 64v^3+48v^2-113v+1 
}
We know that $v=1$ is a solution so $v-1$ must be a root of this polynomial. We can divide this root out of the polynomial in a process that is very similar to normal long division. We start by looking at the cubic term $64v^3$ and ask ourselves what $v-1$ needs to be multiplied by such we end up with this term. The answer is $64v^2$. This tells us that the quadratic polynomial that we want to arrive at has the form 
\eq{
64v^2+X
}, where $X$ are further terms which are yet to be determined. We find them by writing 
\eq{
(64v^2+X)(v-1)=64v^3+48v^2-113v+1, 
}
where we can multiply the left side out to find
\eq{
64v^3-64v^2+X(v-1)=64v^3+48v^2-113v+1
}
and simplify to 
\eq{
X(v-1)=112v^2-113v+1.
}
This equation is now asking the question, what do we need to multiply $v-1$ by to make $112v^2-113v+1.$. We proceed as above and look at the leading term, $112v^2$. this tells us that $X$ must be of the form 
\eq{X=112v+Y,}
where $Y$ is an undetermined constant. Substituting this into out previous condition we find we have
\eqa{
(112v+Y)(v-1)&=&112v^2-113v+1\\
112v^2-112+Y(v-1)&=&112v^2-113v+1\\
Y(v-1)&=&-v+1\\
Y=-1
}
We can now put the solution together from our equations above. The quadratic polynomial we are looking for is  
\eq{
64v^2+X=64v^2+112v+Y=64v^2+112v-1
}
In summary we have shown
\eq{
64v^3+48v^2-113v+1=(v-1)(64v^2+112v-1)
}
On the right-hand side the known solution $v=1$ has been divided out of the rest of the polynomial. So to find the other solutions we now only need to solve the quadratic polynomial 
$64v^2+112v-1$. 

This can now be solved in the normal way, but it's slightly tedious and boring so let's skip this and talk about something more interesting: Did you notice that the polynomial long division is a greedy algorithm? We want to find all the terms of the quadratic polynomial, but in the beginning most are hard to guess. However, we can work out the quadratic term because all subsequent lower-order term that we add cannot interfere with the quadratic. Once we have the quadratic the linear becomes easy, and one we have the linear one, the constant term becomes the easy as well. 