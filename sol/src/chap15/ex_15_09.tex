\exercise{Heterogeneous Expansion}{4}
Consider a network with $N$ nodes and mean degree $z$. Existing links in this network are broken at a rate $1$ per link. At rate $r$ per node a node establishes a link to a randomly chosen partner. 

\subquestion
The dynamics of the degree distribution are captured by the differential equation 
\eq{
\dot{p_k} = -\underbrace{2rp_k}_1 -\underbrace{kp_k}_2 +\underbrace{2rp_{k-1}}_3+\underbrace{(k+1)p_{k+1}}_4    
}
Explain the meaning of the terms 1-4.

\solution 
The terms capture the following processes:

 1. We lose nodes of degree $k$ because they gain an additional link
 
 2. We lose nodes of degree $k$ because they lose one of their links
 
 3.  We gain nodes of degree $k$ because nodes of degree $k-1$ gain a link
 
 4. We gain nodes of degree $k$ because nodes of degree $k+1$ lose a link

\subquestion
Define $G(x)$ as the generating function of the degree distribution. And derive a differential equation that captures the dynamics of $G$.

\solution
We define
\eq{G(x)=\sum p_k x^k}
and write 
\eqa{ 
\dot{G}&=&\sum \dot{p}_k x^k \\
  &=& \sum (-2rp_k - kp_k + 2rp_{k-1} + (k+1)p_{k+1}) x^k \\
   &=& - 2rG - xG' + 2rxG + G' \\
   &=&  2r(x-1)G + (1-x)G' 
}

\subquestion
Consider the steady state of your equation and integrate to obtain $G(x)$. (Use the normalization conditions to fix the constant of integration.) 

\solution
Setting the rate of change to zero yields
\eq{
0= 2r(x-1)G + (1-x)G' 
}
Which we can also write as 
\eq{
G' = 2r G
}
and hence 
\eq{
G(x)=G(0)\exp{2rx} 
}
We don't know $G(0)$ straight away, but we know $G(1)=1$ and hence
\eq{
1=G(0)\exp{2r} 
}
Solving for $G(0)$ yields
\eq{
G(0)=\exp{-2r}
}
which means 
\eq{
G(x)=\exp{2r(x-1)}
}

\subquestion 
We have now found the generating function that the network approaches in the long run. What does it tell you about the network.  

\solution
This is the generating function of en ER graph with $z=2r$.