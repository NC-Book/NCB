\exercise{aSIS model}{4}
We now consider the adaptive SIS model. This model is identical to the network SIS model that we studied before, but additionally susceptible nodes try to distance themselves from infected neighbors. FOr every SI link,t the S node cuts the links at rate $w$. Whenever this happens the S-node establishes a new link to a randomly chosen S node. Hence, the number of links in the model remains constant. 

\subquestion
Write differential equations for the density of infected nodes $[I]$ and a conservation law that governs the density of susceptibles $[S]$. Express the right hand side of the differential equation in terms of node and link densities, don't close it yet. 

\solution
Following exactly the same reasoning as in lecture 14 we arrive at the same result 
\eqa{
\dot{[I]}&=&p[SI]-r[I]\\{}
[S]&=&1-[I]
}
Note that the rewiring does not appear at all as it does not change any node state directly. 

\subquestion
Derive differential equations for the $[SS]$ and $[II]$ link densities and find a conservation law that governs the $[SI]$ link density.

\solution 
Following the reasoning from the lecture we write 
\eqa{
\dot{[SS]}&=&r[I]\frac{[SI]}{[I]}-p[SI]\frac{[SSI]}{[SI]}+w[SI] \\
\dot{[II]}&=&-r[I]\frac{2[II]}{[I]}+p[SI]\left(1+\frac{2[ISI]}{[SI]}\right) 
}
Note that the rewiring term is actually very simple whenever one rewiring event happens it adds one SS. We can simplify the equations to 
\eqa{
\dot{[SS]}&=&r[SI]-p[SSI]+w[SI] \\
\dot{[II]}&=&-2r[II]+p[SI]+2p[ISI] 
}
Finally we use the conservation law for links 
\eq{
[SS]+[SI]+[II]=\frac{z}{2}
}
to write 
\eq{
[SI]=\frac{z}{2}-[SS]-[II]
}
\subquestion Use a pair approximation to close your differential equation system.

\solution
We can use the closures from the lecture
\eqa{
[SSI]&=&\frac{2[SS][SI]}{[S]} \\
2[ISI]&=&\frac{[SI]^2}{[S]} 
}
substitution into the differential equations yields
\eqa{
\dot{[SS]}&=&r[SI]-2p\frac{[SS][SI]}{[S]}+w[SI] \\
\dot{[II]}&=&-2r[II]+p[SI]+\frac{[SI]^2}{[S]} 
}

\subquestion
We now have a dynamical system with 3 dynamical variables $[I],[SS],[II]$ and two auxilliary variables governed by conservation laws. Compute the Jacobian matrix that governs of this system. (Note that you either have to substitute the conservation laws or consider the change of the auxilliary variables in your derivatives.)  

\solution 
In this steady state the Jacobi matrix elements are 
\eqa{
J_{1,1}&=& \frac{\partial}{\partial [I]} [\dot{I}] \\
 &=& \frac{\partial}{\partial [I]} \left( p[SI]-r[I] \right) \\
 &=& -r
}

\eqa{
J_{1,2}&=& \frac{\partial}{\partial [SS]} [\dot{I}] \\
 &=& \frac{\partial}{\partial [SS]} \left( p[SI]-r[I] \right) \\
 &=& p \frac{\partial}{\partial[SS]} [SI] \\
 &=& p \frac{\partial}{\partial[SS]} (z/2-[ss]-[II]) \\
 &=& -p
}
So, this was an example where it mattered that $[SI]$ isn't a dynamical variable, but now governed by a conservation law.  

\eqa{
J_{1,3}&=& \frac{\partial}{\partial [II]} [\dot{I}] \\
 &=& \frac{\partial}{\partial [II]} \left( p[SI]-r[I] \right) \\
 &=& p \frac{\partial}{\partial[II]} [SI] \\
 &=& p \frac{\partial}{\partial[II]} (z/2-[ss]-[II]) \\
 &=& -p
}

\eqa{
J_{2,1}&=& \frac{\partial}{\partial [I]} [\dot{SS}] \\
 &=& \frac{\partial}{\partial [I]} \left( r[SI] - 2p\frac{[SS][SI]}{[S]} + w[SI]\right) \\
 &=&  2p\frac{[SS][SI]}{[S]^2} \frac{\partial [S]}{\partial [I]}  \\
 &=&  -2p\frac{[SS][SI]}{[S]^2} 
}

\eqa{
J_{2,2}&=& \frac{\partial}{\partial [SS]} [\dot{SS}] \\
  &=& \frac{\partial}{\partial [SS]} \left( r[SI] - 2p\frac{[SS][SI]}{[S]} + w[SI]\right) \\
  &=& -r - 2p\frac{[SI]}{[S]}+2p\frac{[SS]}{[S]} - w \\
}
 
\eqa{
J_{2,3}&=& \frac{\partial}{\partial [II]} [\dot{SS}] \\
  &=& \frac{\partial}{\partial [II]} \left( r[SI] - 2p\frac{[SS][SI]}{[S]} + w[SI]\right) \\
  &=& -r +2p\frac{[SS]}{[S]} - w \\
}

\eqa{
J_{3,1}&=& \frac{\partial}{\partial [I]} [\dot{II}] \\
 &=& \frac{\partial}{\partial [I]} \left( p[SI] + p\frac{[SI]^2}{S} - 2r[II]  \right) \\
 &=& \frac{\partial}{\partial [I]} \left( p[SI] + p\frac{[SI]^2}{S} - 2r[II]  \right) \\
 &=&   p\frac{[SI]^2}{S^2} \frac{\partial [S]}{\partial [I]}  \\
 &=&  - p\frac{[SI]^2}{S^2} 
}

\eqa{
J_{3,2}&=& \frac{\partial}{\partial [SS]} [\dot{II}] \\
 &=& \frac{\partial}{\partial [SS]} \left( p[SI] + p\frac{[SI]^2}{S} - 2r[II]  \right) \\
 &=&  -p + p\frac{2[SI]}{S} \frac{\partial [SI]}{\partial [SS]} \\
 &=&  -p - p\frac{2[SI]}{S} 
}

\eqa{
J_{3,3}&=& \frac{\partial}{\partial [II]} [\dot{II}] \\
 &=& \frac{\partial}{\partial [II]} \left( p[SI] + p\frac{[SI]^2}{S} - 2r[II]  \right) \\
 &=& -p - p\frac{2[SI]}{S} - 2r 
}

\subquestion Find the steady state where the epidemic is extinct. Evaluate the Jacobi matrix in that steady state.

\solution
Ecological intuition tells us that in the extinct state there are no infected $[I]=0$ and no links involving infected $[IS]=[II]=0$. Hence
\eqa{
[S]&=&1\\{}
[SS]&=&z/2
}
Substituting back into the equations confirms that this satisfies the conservation laws and the rate of change is zero. 

\eq{
J_{1,1} = -r
}

\eq{
J_{1,2} = -p
}
\eq{
J_{1,3} = -p
}
\eq{
J_{2,1} = -2p\frac{[SS][SI]}{[S]^2} = 0 
} 
\eq{ J_{2,2} = -r -2p\frac{[SI]}{[S]}+2p\frac{[SS]}{[S]} - w = -r+pz-w 
}
 
\eq{
J_{2,3} = -r +2p\frac{[SS]}{[S]} - w = -r +pz -w 
}

\eq{
J_{3,1} = - p\frac{[SI]^2}{S^2} = 0 
}

\eq{
J_{3,2} =  -p - p\frac{2[SI]}{S} = -p 
}

\eq{
J_{3,3} = -p - p\frac{2[SI]}{S} - 2r = -p -2r
}
So in summary
\eq{
{\bf J} = \aveccc{ -r & -p & - p\\ 0 & -r +pz-w & -r +pz-w \\ 0 & -p & -p -2r 
}
}

\subquestion 
You should be able to read off one eigenvalue of the Jacobian straight away. For reasonable choices of parameters this eigenvalue is always negative. The other two eigenvalues are given by a two by two matrix. 
Identify this reduced matrix. 

\solution
Because there is only a diagonal entry in the first column of the Jacobian the diagonal entry is an eigenvalue 
\eq{
\lambda=-r
}
and the remaining eigenvalues are given by the reduced matrix 
\eqn{
\avecc{ -r +pz-w & -r +pz-w \\ -p & -p -2r }
}

\subquestion 
We could now compute the eigenvalues of the remaining 2x2 matrix. But we can also make our life a bit easier. The symmetry of the system (extinct steady state is fixed at zero) tells us that a bifurcation occurring on the extinct state will likely be a transcritical. Hence we expect the epidemic threshold to be a transcritical bifurcation. In this bifurcation one eigenvalue becomes zero and therefore also the determinant becomes zero. Use this to find the epidemic threshold and express it in terms of the critical value of $p$ from which on the disease can invade a healthy network. 

\solution
The determinant is the product of the eigenvalues of a matrix. Hence the determinant is zero if and only if there is a zero eigenvalue. Hence we can find the conditions for a zero eigenvalue to occur by writing 
\eq{
\left| \begin{array}{c c} -r +pz-w & -r +pz-w \\ -p & -p -2r \end{array} \right| =0 
}
Computing the determinant yields 
\eq{
(-r +pz-w)(-p -2r+p) =0 
}
The second term simplifies to $-2r$. Since we are not interested in $r=0$ we can divide by this term which leaves us with 
\eq{
-r +pz-w = 0
}
So the critical value of $p$ is 
\eq{
p = \frac{r+w}{z}
}
One interesting aspect of this equation is that the critical point of the non-adaptive SIS model (e.g. from Ex.~17.1) appears as a special case for $w=0$. 