
% LEVEL 1 %%%%%%%%%%%%%%%%%%%%%%%%%%%%%%%%%%%%%%%%%%%%%%%%%%%%%%%%%%%%%
%%%%%%%%%%%%%%%%%%%%%%%%%%%%%%%%%%%%%%%%%%%%%%%%%%%%%%%%%%%%%%%%%%%%%%

\exercise{Terminology}{1}
Count the number of vertices, edges and components in the network 2 from Fig.~\ref{figMoraviaComponents}. 

\solution
There are six vertices (blue circles), three edges (lines between circles) and three components (connected parts of the network). 


%%% Exercise 2 %%%

\exercise{Number of links}{1}
Consider a fully connected network with 23 nodes. How many links are there in this networks?

% solution %

\solution
We can use the formula
\eq{
K_{\rm max}=\frac{N(N-1)}{2}
}
where $N$ is the number of nodes, so the number of links is $K_{\rm max}=23\cdot 11 = 253$.


%%% Exercise 3 %%%

\exercise{Number of networks}{1}
Compute the number of different networks that can be constructed between 7 labeled nodes.

% solution %

\solution
First we compute the number of links, using the trick from the previous question we get $K_{\rm max}=7 \cdot 3=21$. Every link can be there or not so the total number of different networks is 
\eq{
M=2^{21} = 2097152
}



% LEVEL 2 %%%%%%%%%%%%%%%%%%%%%%%%%%%%%%%%%%%%%%%%%%%%%%%%%%%%%%%%%%%%%
%%%%%%%%%%%%%%%%%%%%%%%%%%%%%%%%%%%%%%%%%%%%%%%%%%%%%%%%%%%%%%%%%%%%%%%


%%% Exercise 4 %%%


\exercise{Return to Moravia\hint{Level 2}{Exercises of level 2 are straightforward applications of the content of the chapter. If you followed the content, these should be easy.}}{2}
 Solve the Moravian city example again. The distances between the cities are the ones given in Eq.~\ref{eqMoraviaMatrix} except that the distance from L to B is now $D_{\rm L,\rm B}=D_{\rm B,\rm L}=100{\rm km}$ instead of 63km. (Perhaps some obstacles exist between these cities that we have to go around) 

% solution %

\solution
The solution network has the edge set 
\eq{
E=\{(\rm L,Z),(B,Z),(O,L),(B,J)\}
}
we find it by
\begin{enumerate}
    \item Try (L,Z) [51km] -- accept
    \item Try (B,Z) [74km] -- accept
    \item Try (L,O) [75km] -- accept
    \item Try (O,Z) [76km] -- reject
    \item Try (B,J) [77km] -- accept 
\end{enumerate}
Of course also just drawing the solution network is also an acceptable answer to this type of question.


%%% Exercise 5 %%%

\exercise{Another power grid}{2}
Construct a power grid between 4 cities $V=\{ {\rm A},{\rm B},{\rm C},{\rm D}\}$. Where the kilometer of lines to connect a given pair of cities is distances given by the distance matrix
\eqs{
  {\bf D}=\left(\begin{array}{c c c c} 
    0  & 17 & 23 & 9\\
    17 & 0  & 18 & 13\\
    23 & 18 & 0  & 27\\
    9  & 13 & 27 & 0
   \end{array} \right).
  }
  
% solution %

\solution
We do the following:
\begin{enumerate}
\item Try (A,D) [9km] -- accept
\item Try (B,D) [13km] -- accept
\item Try (A,B) [17km] -- reject
\item Try (B,C) [18km] -- accept  
\end{enumerate}
So, the resulting edge set is 
\eq{
E=\{{(\rm A,\rm D),(\rm B,\rm D),(\rm B,\rm C)}\}.
}


%%% Exercise 6 %%%

\exercise{Larger abstract example\hint{Drawing nets}{If you draw unknown networks like this one it is best to arrange nodes in a circle.}}{2}
Construct the minimal spanning tree in a network where the weight of links is given by
\eq{
{\bf D}=\left(\begin{array}{c c c c c c} 
0  & 8  & 1  & 14 & 4  & 5  \\
8  & 0  & 7  & 12 & 9  & 10  \\
1  & 7  & 0  & 11 & 3  & 2  \\
14 & 12 & 11 & 0  & 15 & 13  \\
4  & 9  & 3  & 15 & 0  & 6  \\
5  & 10 & 2  & 13 & 6  & 0  \\   
\end{array} \right).
}
To reduce the tediousness the distances in this exercise have been chosen as 1,2,3, and so on. 

% solution %

\solution
Using Kruskal's algorithm we do the following:
\begin{enumerate}
\item Try (1,3) [1] -- accept 
\item Try (3,6) [2] -- accept
\item Try (3,5) [3] -- accept
\item Try (1,5) [4] -- reject
\item Try (1,6) [5] -- reject
\item Try (5,6) [6] -- reject
\item Try (2,3) [7] -- accept
\item Try (1,2) [8] -- reject
\item Try (2,5) [9] -- reject
\item Try (2,6) [10] -- reject
\item Try (3,4) [11] -- accept
\end{enumerate}
Hence, the edge set of the solution is 
\eq{
E=\{(1,3),(3,6),(3,5),(2,3),(3,4) \}
}



% LEVEL 3 %%%%%%%%%%%%%%%%%%%%%%%%%%%%%%%%%%%%%%%%%%%%%%%%%%%%%%%%%%%%%
%%%%%%%%%%%%%%%%%%%%%%%%%%%%%%%%%%%%%%%%%%%%%%%%%%%%%%%%%%%%%%%%%%%%%%%


%%% Exercise 7 %%%

\exercise{Ethernet\hint{Level 3}{Exercises of level 3 are variations on the content of the chapter, but they can be a bit longer and might contain twists that need some thought or take the methods to a different context. This is the level that we eventually want to reach.}}{3}
 I want to install a wired network connection in my house. My internet connection is via a router that sits in the cellar, and I want to connect the bedroom, living room, and the kitchen. I don't mind installing network switches in these rooms such that a room can get network access from any other room that has network access. To connect the cellar to the kitchen I would need 6m of cable, to the living room its 8m and to the bed room its 12m. But I could connect the kitchen to the bedroom with just 3m, and to the living room with 7m. Finally the bedroom could be connected to the living room with 4m of cable. What length of cable do I need to buy? 

% solution %

\solution
Here the first step is to extract the distance from the text, they are 
\begin{itemize}
\item (C,K) - 6m
\item (C,L) - 8m
\item (C,B) - 12m
\item (K,B) - 3m
\item (K,L) - 7m
\item (B,L) - 4m
\end{itemize}
where the nodes are C: Cellar, K: Kitchen, L: Living Room, B: Bedroom. Now we can apply our alogrithm
\begin{enumerate}
\item Try (K,B) [3m] -- accept
\item Try (B,L) [4m] -- accept
\item Try (C,K) [6m] -- accept
\end{enumerate}
We connect (K,B),(B,L),(C,K) and we are already done. I only need 13m of cable.


%%% Exercise 8 %%%

\exercise{Another network in Moravia}{3}
Let's revisit the Moravian example. This time we assume that when we get to work there is already a power line from B to Z. Find which additional lines need to be built such that all cities are connected and the length of additional lines built is minimal. 

% solution %

\solution
We can apply Kruskal's algorithm, but line (B,Z) is already there (or alternatively, we could say it has cost 0). So what we do is
\begin{enumerate}
\item Place (B,Z)
\item Try (L,Z) [52km] -- accept
\item Try (L,B) [63km] -- reject
\item Try (L,O) [75km] -- accept
\item Try (O,Z) [76km] -- reject
\item Try (B,J) [77km] -- accept
\end{enumerate}
So, the final edge set is 
\eq{
E= \{{\rm (B,Z),(L,Z),(L,O),(B,J)}\}
}


%%% Exercise 9 %%%

\exercise{A tale of two cities}{3}
Consider the Moravian power grid again. This time we start again with an empty graph, but there are two power plants, which are located in B and Z. Each of the plants has the capacity to supply the whole region. Find the lines that need to be built such that each city can get power from one of the power plants and the length of lines built is minimal. 

% solution %

\solution
We apply a variant of Kruskal's algorithm: In addition to rejecting any link that does not reduce the number of components, we also reject any link that would connect the two components in which B and Z are located.
\begin{enumerate}
\item Try (L,Z) [51km] -- accept
\item Try (L,B) [63km] -- reject (puts B and Z in same component)
\item Try (B,Z) [74km] -- reject (puts B and Z in same component)
\item Try (L,O) [75km] -- accept
\item Try (O,Z) [76km] -- reject (does not reduce number of components)
\item Try (J,B) [77km] -- accept
\end{enumerate}
Now everybody is connected to one of the power plants. The final edge set is 
\eq{
E= \{{\rm (L,Z),(L,O),(J,B)}\}
}


%%% Exercise 10 %%%

\exercise{Pandemic Rideshare}{3}
During a pandemic Dave and Peter are driving home. Their friend Bob asks them if they can give him a lift. However, the two are slightly worried because such close contacts allow the disease to spread and there are many asymptomatic infections. 

\subquestion
Bob argues that adding another person only increase the group size by 50\%, but by what factor has the number of contacts increased?

\solution
With three people in the car we are creating 3 contacts (Dave-Peter, Peter-Bob, Dave-Bob) with 2 people it would be only one. So while the group size is only 50\% greater, the number of contacts grows by 200\%. (Note that the people in the car form a fully connected network, so we could use the formula from the chapter, but due to the small numbers we hardly need it.)

\subquestion
Actually Dave was planning to have his birthday party with 20 people on Friday night. Now he is wondering how many contacts that would create?

\solution
Again we assume a fully connected network. In the birthday party the number of contacts is 
\eq{
\frac{20\cdot19}{2} = 190
}
So the 20 people meeting at the party cause 190 contacts, that's quite alot.


% LEVEL 4 %%%%%%%%%%%%%%%%%%%%%%%%%%%%%%%%%%%%%%%%%%%%%%%%%%%%%%%%%%%%%
%%%%%%%%%%%%%%%%%%%%%%%%%%%%%%%%%%%%%%%%%%%%%%%%%%%%%%%%%%%%%%%%%%%%%%%


%%% Exercise 11 %%%

\exercise{Pruning networksv\hint{Level 4}{On level 4 you  develop your own methods, this is tough but can be very rewarding if you can work it out.
}}{4}
Develop your own algorithm for finding minimal spanning trees. In contrast to Kruskal's algorithm start with a fully-connected network and then remove links until only the minimum spanning tree is left. Make sure that your algorithm always yields the optimal result.

% solution %

\solution
To find the desired algorithm we need to think about the problem differently. In Kruskal's we structured our thinking around the number of components, we needed to bring it down. When we start from the fully connected network the number of components is already one, so what do we look at instead? 

Actually our goal is to remove links. The fully connected network will have a number of links, lets call this number $M$. In analogy to our previous reasoning we can say that the fully connected network is the optimal network with $M$ links. 
Now it is easy to find the optimal network with $M-1$ links, we simply remove the most expensive link. We then find the optimal network with $M-2$ links by removing the second most expensive link, whose removal does not increase the number of components and so on.  

So our algorithm is the following: Start with the fully connected network. Then consider the links one by one in the order of reverse cost. If removing the link increases the number of components then we retain the link otherwise we remove it.   
For Moravia this means:
\begin{enumerate}
\item Consider (O,J) [198km] -- remove 
\item Consider (Z,J) [151km] -- remove
\item Consider (B,O) [137km] -- remove
\item Consider (L,J) [121km] -- remove
\item Consider (B,J) [77km] -- retain
\item Consider (O,Z) [76km] -- remove
\item Consider (O,L) [75km] -- retain
\item Consider (B,Z) [74km] -- remove
 \end{enumerate}
The final step leaves us with a tree, so we can't remove any more links. As before we find that the optimal power grid has the edge set $E=\{{\rm(O,L),(L,Z),(B,L),(B,J) }\}$



% LEVEL 5 %%%%%%%%%%%%%%%%%%%%%%%%%%%%%%%%%%%%%%%%%%%%%%%%%%%%%%%%%%%%%
%%%%%%%%%%%%%%%%%%%%%%%%%%%%%%%%%%%%%%%%%%%%%%%%%%%%%%%%%%%%%%%%%%%%%%%


%%% Exercise 12 %%%

\exercise{Counting graphs\hint{Level 5}{Sometimes there will be a question of Level 5. These are unsolved problems. If you solve one of them, make sure to publish your solution.}}{5}
When we derived the formula for the number of network that can be constructed with a given number of nodes, we emphasized that the nodes are labelled. In an unlabelled graph the nodes are indistinguishable, so for example all networks that contain five nodes and  one link are actually the same network, no matter which two nodes the link connects, To warm up find the number of networks that can be constructed between 3 unlabelled nodes (The answer is 4). Then try four nodes. Find a general formula that works for any number of nodes. 

% solution %

\solution
Between four nodes 11 different unlabeled graphs can be constructed. 1: no links, 2: one link, 3: two links connecting to the same node, 4: two links that don't share a node, 5: three links in a line, 6: three links in a triangle, 7: three links, all connecting to the same node, 8: four links in a square, 9: four links in a triangle with a handle, 10: five links, 11: six links.

This is an unsolved problem. If we had a solution it would make many it would make significant progress in many important applications possible.

\solutionend


